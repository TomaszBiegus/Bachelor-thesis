%%%%%%%%%%%%%%%%%%%%%%%%%%%%%%%%%%%%%%%%%%%%%%%%%%%%%%%%%
% Niniejszy plik przedstawia przykładowy skład 
% pracy dyplomowej na Wydziale Matematyki PWr. 
% 
% Autorzy: 
% Damian Fafuła
% Michał Kijaczko
% Jakub Michalczak
% Maciej Miśta
% Dagmara Nowak
% Tomasz Skalski
% Wojciech Słomian
%
%% Data utworzenia: 8.05.2018
% Numer wersji: 1
%
% Poniższą formatkę można rozpowszechniać i edytować 
% pod warunkiem zachowania numeru wersji, 
% informacji o autorach i dodaniu informacji 
% o wprowadzonych zmianach.
%
%%%%%%%%%%%%%%%%%%%%%%%%%%%%%%%%%%%%%%%%%%%%%%%%%%%%%%%%%
% Domyślną opcją jest: praca magisterska, język polski.
% W przypadku pracy pisanej w języku angielskim dodajemy 
% opcję [english].
% Dla pracy licencjackiej dodajemy opcję [licencjacka].
% Dla pracy inżynierskiej dodajemy opcję [inzynierska].
% Dopuszczalne są podwójne opcje, np. [licencjacka, english].
% Opcje dodajemy w kwadratowym nawiasie przy \documentclass.
%
%
%%%%%%%%%%%%%%%%%%%%%%%%%%%%%%%%%%%%%%%%%%%%%%%%%%%%%%%%%
\documentclass[licencjacka]{pwr_wmat_praca_dyplomowa}

\usepackage{enumitem}
\usepackage{xcolor}
\usepackage{mathtools}
%%%%%%%%%%%%%%%%%%%%%%%%%%%%%%%%%%%%%%%%%%%%%%%%%%%%%%%%%
%              DANE DO PRACY
%
% W przypadku pracy dyplomowej w języku angielskim nie jest konieczne 
% wypełnianie pól: \tytul{}, \kierunek{}, \specjalnosc{}, 
%                  \streszczenie{}, \slowakluczowe{}.
%%%%%%%%%%%%%%%%%%%%%%%%%%%%%%%%%%%%%%%%%%%%%%%%%%%%%%%%%
%
% Imię i nazwisko autora
\autor{Tomasz Biegus}
%
% Tytuł pracy dyplomowej 
\tytul{Dynamika odwzorowań trójkątnych} 
\tytulang{Tytuł pracy dyplomowej w języku angielskim}
%
% Tytuł / stopień / imię i nazwisko opiekuna
\opiekun{dr inż. Dawid Huczek}
%
% Kierunek studiów wybieramy spośród następujących:
% 1) Matematyka
% 2) Matematyka i Statystyka
% 3) Matematyka stosowana
\kierunekstudiow{Matematyka}
%
% Kierunek studiów po angielsku wybieramy spośród następujących:
% 1) Mathematics
% 2) Mathematics and Statistics
% 3) Applied Mathematics
\kierunekstudiowang{Mathematics}
%
% Specjalność wybieramy spośród następujących: 
% KIERUNEK: Matematyka
% 1) Matematyka teoretyczna,
% 2) Statystyka matematyczna,
% 3) Matematyka finansowa i ubezpieczeniowa,
%
% KIERUNEK: Matematyka i Statystyka
% 4) Matematyka,
% 5) Statystyka i analiza danych, 
%
% 6) -- (w przypadku braku specjalizacji).
\specjalnosc{Matematyka teoretyczna} 
%
% Specjalność w języku angielskim wybieramy spośród następujących:
% KIERUNEK: Matematyka
% 1) Theoretical Mathematics,
% 2) Mathematical Statistics,
% 3) Financial and Actuarial Mathematics,
%
% KIERUNEK: Matematyka i Statystyka
% 4) Mathematics,
% 5) Statistics and Data Analysis,
%
% KIERUNEK: Applied Mathematics
% 6) Financial and Actuarial Mathematics, 
% 7) Mathematics for Industry and Commerce,
% 8) Computational Mathematics,
% 9) Modelling, Simulation and Optimization.
%
% 10) -- (w przypadku braku specjalizacji).
\specjalnoscang{Theoretical Mathematics} 
%
% Krótkie streszczenia po polsku i angielsku
% - nie dłuższe niż 530 znaków.
\streszczenie{Celem niniejszej pracy jest zaprezentowanie dowodu pewnego twierdzenia dotyczącego chaotycznych układów dynamicznych na przestrzeniach produktowych z czasem dyskretnym. Praca zawiera twierdzenie wraz z dowodem, pomocniczymi twierdzeniami, lematami i niezbędnymi definicjami. Omawiane są takie pojęcia jak entropia topologiczna, chaos w sensie Devaneya, klasa ciągłych odwzorowań trójkątnych na przestrzeni produktowej, rozszerzenie do odwzorowania trójkątnego.}
\streszczenieang{The purpose of the thesis is to present a proof of a theorem regarding chaotic dynamic systems on product spaces with a discrete time. The work contains a theorem along with proof, auxiliary statements, lemmas and necessary definitions. Issues such as topological entropy, chaos in the sense of Devaney, a class of continuous triangular maps on the product space, extension to triangular map are discussed.}
%
% Podajemy najważniejsze słowa kluczowe po polsku i angielsku
% - w obu przypadkach, nie więcej niż 150 znaków.
\slowakluczowe{chaos,
entropia topologiczna, 
odwzorowania trójkątne, 
układy dynamiczne.}  
\slowakluczoweang{chaos,
topological entropy,
triangular maps,
dynamical systems.}
%
%
%%%%%%%%%%%%%%%%%%%%%%%%%%%%%%%%%%%%%%%%%%%%%%%%%%%%%%%%%
% Definicje, lematy, twierdzenia, przykłady i wnioski
% Komendy wywołujące twierdzenia, definicje, itd., 
% czyli 'theorem', 'definition', 'corollary', itd., 
% można zmienić wedle uznania.
\theoremstyle{plain}
\newtheorem{theorem}{Twierdzenie}
\numberwithin{theorem}{chapter}
\newtheorem{lemma}[theorem]{Lemat} 
\newtheorem{corollary}[theorem]{Wniosek}
\newtheorem{fact}[theorem]{Fakt}
\theoremstyle{definition}
\numberwithin{theorem}{chapter}
\newtheorem{definition}[theorem]{Definicja} 
\newtheorem{example}[theorem]{Przykład}
\newtheorem{note}[theorem]{Uwaga}
%%%%%%%%%%%%%%%%%%%%%%%%%%%%%%%%%%%%%%%%%%%%%%%%%%%%%%%%%


%%%%%%%%%%%%%%%%%%%%%%%%%%%%%%%%%%%%%%%%%%%%%%%%%%%%%%%%%
%%%%%%%%%%%%%%%%%%%%%%%%%%%%%%%%%%%%%%%%%%%%%%%%%%%%%%%%%
\begin{document}
\frontmatter
\maketitle
\mainmatter
\tableofcontents
%\listoffigures
%\listoftables



%\chapter{Rozdział pierwszy}
%Tabela \ref{tab:przykladowa} przedstawia przykładową tabelę. Do tworzenia tabeli służą m.in. środowiska \texttt{tabular} oraz \texttt{table}. Istnieje możliwość numeracji dwustopniowej, gdzie pierwsza cyfra oznacza numer rozdziału, a druga – kolejny numer tabeli w tym rozdziale. Tytuł powinien znajdować się centralnie nad tabelą, $12$ pkt odstępu od tekstu zasadniczego nad i pod tabelą wraz z tytułem. Jeśli tabela jest cytowana – należy podać centralnie pod tabelą źródło jej pochodzenia, np. opracowanie własne, opracowano na podstawie danych z GUS.
%\begin{table}[ht]
%\caption{Podstawowa Tabela}
%\centering
%\begin{tabular}{ccc}
%\hline
%\hline                       
%Państwo & PKB (w milionach USD )& Stopa bezrobocia  \\  [0.5ex] 
%\hline 
%Stany Zjednoczone & 75 278 049 & 4,60\%  \\
%Chiny & 11 218 281 & 4,10\%   \\
%Japonia & 4 938 644 & 3,10\%  \\
%Niemcy & 3 466 639 & 6,00\%   \\
%Wielka Brytania & 2 629 188 & 4,60\%  \\ [1ex]  
%\hline 
%\end{tabular}
%\caption*{\textit{Źródło: opracowanie własne}}
%\label{tab:przykladowa} 
%\end{table}
%
%Do cytowania używamy komendy \texttt{cite}. W nawiasie klamrowym podajemy klucz, którego użyliśmy w pliku \emph{bibliografia.bib}. Przykład: \cite{einstein} lub \cite[chap. 2]{latexcompanion}.
%
%\section{Podrozdział pierwszy}
%
%\begin{table}[H]
%\caption{Podstawowa Tabela}
%\centering
%\begin{tabular}{ccc}
%\hline
%\hline                       
%Państwo & PKB (w milionach USD )& Stopa bezrobocia  \\  [0.5ex] 
%\hline 
%Stany Zjednoczone & 75 278 049 & 4,60\%  \\
%Chiny & 11 218 281 & 4,10\%   \\
%Japonia & 4 938 644 & 3,10\%  \\
%Niemcy & 3 466 639 & 6,00\%   \\
%Wielka Brytania & 2 629 188 & 4,60\%  \\ [1ex]  
%\hline 
%\end{tabular}
%\caption*{\textit{Źródło: opracowanie własne}}
%\label{tab:przykladowa2} 
%\end{table}
%
%\section{Podrozdział drugi}
%
%Rysunki do pracy dyplomowej należy wstawiać w sposób podobny do wstawiania tabel, z~zasadniczą różnicą polegającą na tym, że podpis powinno umieszczać się centralnie pod rysunkiem, a nie powyżej niego. Numeracja i sposób cytowania pozostają bez zmian, przy czym tabele i rysunki nie mają numeracji wspólnej, np. po Tabeli \ref{tab:przykladowa2} występuje Rysunek \ref{rys1} (o ile jest to pierwszy rysunek rozdziału pierwszego), a nie Rysunek $1.3$.
%
%\begin{figure}[ht]
%
%\centering
%                     
%\includegraphics[scale=0.27]{logo_w13.jpg}
%\caption{Podstawowy Rysunek}\label{rys1}
%\end{figure}
%\label{rys:przykladowy} 
%
%
%
%\chapter{Definicje, lematy, twierdzenia, przykłady i wnioski}
%Definicje, lematy, twierdzenia, przykłady i wnioski piszemy w pracy tak:
%\begin{definition}[Martyngał]
%Tu piszemy treść definicji martyngału.
%\end{definition}
%\begin{lemma}[]% w nawiasie kwadratowym można napisać jego nazwę
%Tu piszemy treść lematu.
%\end{lemma}


% MOJE ================================================
{\backmatter \chapter{Wstęp}}
\subsubsection{Motywacje}
Studia matematyczne podjąłem późno, już po ukończeniu innych kierunków. Od zawsze jednak matematyka pociągała mnie jako klucz do zrozumienia pozostałych dziedzin, jako narzędzie pozwalające dostrzec lepiej i głębiej urodę przyrody. Wykorzystując matematykę, możemy uchwycić reguły, wzorce i schematy które opisują niewyobrażalną złożoność świata w którym żyjemy. Matematykę postrzegam jako klamrę spinające wszystkie dziedziny intelektualnej działalności człowieka, począwszy od fizyki, poprzez chemię, biologię, aż po psychologię czy nauki społeczne. Decyzję o rozpoczęciu studiów matematycznych podjąłem w lecie siedząc nad rzeką i przyglądając się hipnotyzującym zawirowaniom na powierzchni wody płynącej w rzece Odrze przepływającej przez Wrocław, oraz leniwemu ruchowi fraktalnych kształtów chmur na błękitnym niebie. Byłem wówczas absolwentem wyższej uczelni, w którym wciąż żyło pragnienie poznawania głębiej wszystkich tych pięknych zjawisk, które nas otaczają. Byłem również świeżo po lekturze inspirującej popularnonaukowej książki Jamesa Gleicka zatytułowanej ,,Chaos'' \cite{gleick1987chaos}...

\subsubsection{Cel pracy}
Celem niniejszej pracy jest przedstawienie wyniku twierdzenia zawartego w pracy \cite{balibrea2003topological} wraz z uzupełnieniem fragmentów, w których autorzy powoływali się na inne źródła, bądź pomijali dowody faktów uznawanych przez nich za oczywiste, a które studentowi matematyki oczywiste wydawać się nie muszą. Twierdzenie to dotyczy chaotycznych układów dynamicznych z czasem dyskretnym, gdzie przestrzenią fazową jest iloczyn kartezjański zwartej przestrzeni metrycznej i domkniętego odcinka jednostkowego.

\subsubsection{Historia badań nad układami dynamicznymi}
Historia rozwoju nauki nierozerwalnie wiąże się z pojęciem ruchu. Bez ruchu nie ma sensu mówić o upływie czasu, zachodzeniu procesów fizycznych chemicznych czy też biologicznych, czyli tych wszystkich rzeczach wokół których koncentrują się badania naukowe. Rozważania nad ruchem są zapewne tak stare jak gatunek ludzki. Pisane świadectwa zadawania sobie takich pytań znajdujemy już w dziełach starożytnych. Heraklit z Efezu (ur. ok. 540 p.n.e., zm. ok. 480 p.n.e.) miał powiedzieć ,,pantha rei'' co znaczy ,,wszystko płynie'' dostrzegając procesy nieustannych zmian w otaczającym nas świecie. Zenon z Elei rozważał paradoksy ruchu, Arystoteles sformułował swoje prawa ruchu. Następnie Izaak Newton, jak sam to ujął ,,stojąc na barkach gigantów'' Galileusza, Tychona Brahego i Johannesa Keplera, zapoczątkował rozkwit współczesnych nauk empirycznych formułując w 1687r. zasady dynamiki i (równolegle z Gottfriedem Wilhelmem Leibnizem) dając początek rachunkowi różniczkowemu. Rachunek różniczkowy dał potężne narzędzie do modelowania rzeczywistości, pozwalając na przykład na przewidywanie konfiguracji ciał niebieskich na niebie w dalekiej przyszłości. Niesamowity rozwój nauki w wieku XIX skłaniał ludzi do pokładania coraz większej ufności w możliwość absolutnie pewnego przewidywania stanów dowolnych układów. Rozbudzało to wyobraźnię i nadzieje na dokładne przewidywanie pogody, rozwoju organizmów żywych, rozprzestrzeniania się chorób a być może nawet zachowania ludzi. Na przełomie wieków XIX i XX pojawiły się potężne pęknięcia, niwecząc te ambitne wyobrażenia. Nastąpiły trzy odkrycia, wskazujące na immanentne ograniczenia w możliwościach badania świata. Były to: twierdzenie G\"odla, zasada nieoznaczoności Heisenberga i odkrycie chaosu deterministycznego. Tego ostatniego dotyczy niniejsza praca. Historia badań nad układami chaotycznymi sięga XIX wieku, kiedy to wielki matematyk francuski Henri Poincar\'e postanowił zmierzyć się z wielkim pytaniem: ,,Czy układ słoneczny jest stabilny?'', nie udało mu się udzielić odpowiedzi, ale metody które rozwinął przyczyniły się do powstania nowej gałęzi matematyki, nazywanej teorią układów dynamicznych. Przełomową chwilą dla nauki o chaosie było okrycie Lorenza - Meteorologa. Pokazał on, że uproszczony układ trzech równań różniczkowych opisujących ruch powietrza w atmosferze, prowadzi do chaosu. To znaczy, że orbity rozwiązań tych równań dążą w przestrzeni fazowej do atraktora, który posiada niezwykle złożoną strukturę, która tłumaczy dlaczego przewidywanie pogody na dowolnie długie okresy wprzód nie jest i co ważniejsze nigdy nie będzie możliwe.

Opisywane powyżej pory zagadnienia opierały się na równaniach różniczkowych. Czas w tym modelu jest ciągły, to znaczy, rozwiązanie układu równań różniczkowych jest funkcja czasu postaci $f: \mathbb{R} \rightarrow X,$ gdzie $X$ jest przestrzenią wszystkich możliwych stanów układu, tak zwaną przestrzenią fazową. Matematycy badają jednak również układy w których znamy stan układu tylko w dyskretnych chwilach czasu i dysponujemy formułą określającą jaki będzie kolejny stan układu jeżeli znamy stan obecny. Przykładem zastosowania takiego podejścia do badań fizycznych są przekroje Poincar\'ego stosowane przez niego podczas prób rozwiązania wspomnianego wcześniej problemu stabilności układu słonecznego. W takim przypadku zamiast równań różniczkowych mamy odwzorowanie $f: X \rightarrow X,$ gdzie stan układu w kolejnych chwilach czasu przyjmuje wartości $x_0, f(x_0), f^2(x_0), f^3(x_0), \ldots.$





\subsubsection{Chaos deterministyczny}
Pojęcie chaosu deterministycznego ma swój początek w naukach empirycznych, w szczególności w fizyce, która zajmuje się badaniem układów podlegających zmianom w czasie. Wprowadza się pojęcie przestrzeni fazowej. Jest to wielowymiarowa przestrzeń, której punktom odpowiadają stany układu. Przykładowo stan prostego wahadła możemy przedstawić jako punkt w przestrzeni dwuwymiarowej, gdzie pierwsza współrzędna odpowiada wychyleniu na przykład w radianach, a druga momentowi pędu wahadła. Gdy wahadło się kołysze, zmienia się położenie punktu reprezentującego jego stan. Ewolucję układu możemy badać analizując trajektorie w tej przestrzeni. Bardziej skomplikowanym przykładem jest analiza ruchu białka, gdzie położenia i pędy każdego z jego atomów są współrzędnymi w przestrzeni stanów. Zauważmy, że do pełnego opisu stanu białka potrzebujemy $6 \cdot N$ współrzędnych, gdzie $N$ jest liczbą atomów w białku (liczba 6 bierze się stąd, że musimy znać po trzy współrzędne dla położeń i pędów). W takim przypadku operujemy w bardzo wysoko wymiarowej przestrzeni fazowej. Ruch w przestrzeni fazowej jest matematyczną abstrakcją i nie należy go utożsamiać z fizycznym ruchem badanego obiektu. Jeżeli w pewnym momencie układ powróci do punktu w którym już wcześniej się znajdował, to od tego momentu będzie powtarzał się cykl, zjawisko jest periodyczne, a więc w pełni przewidywalne. Okazuje się jednak, że są zjawiska, które nie osiągają nigdy stanu w którym już kiedyś się znajdowały, powracając być może dowolnie blisko do stanów już odwiedzonych, ale jednak nie wpadając w cykl. Ruch taki, mimo że w pełni deterministyczny, wydaje się być losowy i jest nieprzewidywalny. Intuicyjnie zjawisko chaotyczne możemy rozumieć następująco. Jeżeli rozważymy dwa układy rządzone przez jednakowe prawa to jeżeli wystartujemy z bliskich, ale nie jednakowych, stanów początkowych, to po skończonym czasie układy te staną się odległe w przestrzeni fazowej. Oznacza to tyle, że startując z bardzo podobnego położenia po jakimś czasie oba układy będą się zachowywały zupełnie inaczej.

Kiedy dysponujemy pojęciem przestrzeni fazowej i trajektorii punktu w tej przestrzeni, możliwe staje się odejście od fizycznych zjawisk, które były modelowane i sprowadzenie zagadnienia do czystej matematyki. 

Chaos okazuje się niełatwym do zdefiniowania pojęciem i nie ma jednej powszechnie przyjętej definicji chaosu. W matematyce zaproponowano ich kilka. 
Li-Yorke, Von Neumann, Birkhoff, Smale, Szarkowski, Devaney






\chapter{Wprowadzenie}

\section{Definicje}



\begin{enumerate}

\item podciąg
TODO: Dopisać do sekcji definicji definicję punktu skupienia ciągu.

\item zbieżność jednostajna (topologia zbieżności jednostajnej (pojawia się w definicji metryki na przestrzeni odwzorowań trójkątnych))



\item \textcolor{red}{baza topologii w przestrzeni zwartej (kule o środkach w ośrodku i promieniach wymiernych - potrzebne w dowodzie ogólnym patrz TODO na czerwono)}
\item punkt izolowany w przestrzeni metrycznej


\item Ciągłość funkcji na przestrzeni metrycznej 
\item (zbiór $C(X)$)






\item entropia topologiczna - dowod rownowaznosci (ksiazka misiurewicz combinatorial dynamics and entropy in dimension one oraz ksiazka ruette rozdzial 4)

%\item topologiczna tranzytywnosc, jest
%\item wrazliwosc na warunki poczatkowe, jest
%\item RODZAJE CHAOSU (Devaneya, jest
%\item chaos Li Yorka), jest


\item napisac ze chaotycznosc odwzorowania rozumiem przez chaotycznosc odpowiedniego ukladu dynamicznego

\item zbiór rezydualny
\item separable, second category (czyli 1 i 2 kategoria bairea)
\item twierdzenie baire'a dla przestrzeni metrycznych zupełnych napisac
\item g-delta
\item topologia zbieżności jednostajnej? Potrzebne do lematow 1.60 i 1.61 jezeli nie usune fragmentu sformuowania tych lematow




\item odwzorowania trójkątne, zbior $C_\triangle(X \times I)$

\end{enumerate}

% Przyklad cytowania z numerem strony
%\begin{definition}[Metryka]
%TODO
%\cite[s.~103]{kuratowski1977wstep}
%\end{definition}


\subsection{Definicje i fakty ogólne}






\subsection{Definicje dotyczące zwartych przestrzeni metrycznych}

\begin{definition}[Metryka]
Metryką na zbiorze $X$ nazywamy funkcję $\rho : X \times X \longrightarrow \mathbb{R}_+ \cup \{0\}$ spełniającą następujące warunki:
\begin{enumerate}
\item $\forall_{x,y \in X}: \rho(x,y)=0 \iff x=y$,
\item $\forall_{x,y \in X}: \rho(x,y) = \rho(y,x)$,
\item \label{nierownosc_trojkata} $\forall_{x,y,z \in X}: \rho(x,y) \leq \rho(x,z) + \rho(z,y)$.
\end{enumerate}
Warunek \ref{nierownosc_trojkata} nazywany jest zwykle nierównością trójkąta.

\end{definition}

\begin{definition}[Przestrzeń metryczna]
Przestrzenią metryczną nazywamy parę $(X, \rho)$, gdzie $X$ jest zbiorem a $\rho$ zdefiniowaną na nim metryką. Czasami, tam gdzie nie będzie prowadziło to do nieporozumień, przestrzeń metryczną $(X, \rho)$ będziemy oznaczać przez samo $X$.
\end{definition}

\begin{definition}[Kula otwarta]
Kulą otwartą w przestrzeni metrycznej $(X, \rho)$ nazywamy zbiór: $K(s, r) = \{x \in X: \rho(s, x) < r\}$.
Punkt $s$ nazywamy wówczas środkiem kuli $K$, a $r \in \mathbb{R}_+ \cup \{0\}$ jej promieniem.
\end{definition}

\begin{definition}[Zbiór otwarty]
Zbiór $A$ w przestrzeni metrycznej $(X, \rho)$ nazywamy otwartym wtedy i tylko wtedy gdy, dla każdego $a \in A$ istnieje kula otwarta o środku w $a$, zawierająca się w $A$.
\end{definition}

\begin{lemma}[Kula otwarta w przestrzeni metrycznej jest zbiorem otwartym]\label{kula_otwarta_jest_zbiorem_otwartym}
W przestrzeni metrycznej $(X, \rho)$, każda kula otwarta jest zbiorem otwartym.
\end{lemma}

\begin{proof}
Niech $(X, \rho)$ będzie przestrzenią metryczną. Weźmy dowolną kulę otwartą $K = K(s, r) \subseteq X$. Weźmy teraz dowolny punkt $k \in K$. Wiemy, że $\rho(s, k) < r$, czyli $r - \rho(s, k) = \epsilon > 0$. Zatem, jeżeli weźmiemy $r_k = \frac{\epsilon}{2}$ to $K_2 = K(k, r_k) \subset K$, ponieważ 

\begin{equation}
\begin{aligned}
\forall_{a \in K_2} \\
& \rho(s, a) \leq \rho(s, k) + \rho(k, a) \leq \rho(s, k) + r_k \\
& = \rho(s, k) + \frac{\epsilon}{2} = \rho(s, k) + \frac{r}{2} - \frac{\rho(s, k)}{2} \\
& = \frac{\rho(s, k)}{2} + \frac{r}{2} < r,
\end{aligned}
\end{equation}
czyli $\forall_{a \in K_2} \, a \in K.$ Z dowolności $k$ otrzymujemy, że kula $K$ jest otwarta a z dowolności wyboru kuli $K$ otrzymujemy tezę.
\end{proof}


\begin{definition}[Otoczenie punktu]
W przestrzeni metrycznej $X$ zbiór $V \subseteq X$ nazywamy otoczeniem punktu $x$, jeżeli istnieje zbiór otwarty $U \subseteq X,$ taki że $x \in U \subseteq V$.
\end{definition}

\begin{definition}[Otoczenie zbioru]
W przestrzeni metrycznej $X$ zbiór $V \subseteq X$ nazywamy otoczeniem zbioru $A$, jeżeli istnieje zbiór otwarty $U \subseteq X,$ taki że $A \subseteq U \subseteq V$.
\end{definition}

%\begin{definition}[Sąsiedztwo punktu]
%Sąsiedztwem punktu $x \in X$ nazywamy każdy zbiór $V_x = V \setminus \{x\}$, gdzie $V$ jest pewnym otoczeniem punktu $x$.
%\end{definition}


\begin{definition}[Wnętrze zbioru]
W przestrzeni metrycznej, wnętrzem zbioru $A$ nazywamy zbiór wszystkich punktów, które należą do $A$ wraz z pewnym swoim otoczeniem.
\end{definition}

\begin{definition}[Domknięcie zbioru]
Domknięciem zbioru $A \subset X$ nazywamy zbiór $\{x \in X: \forall_{r>0} \, K(x, r) \cap A \neq \emptyset\}$. Domknięcie zbioru $A$ oznaczamy przez $\bar{A}$ lub $\textrm{Cl}(A)$.
\end{definition}

\begin{definition}[Zbiór domknięty]
Zbiór $A$ jest domknięty, gdy $A = \textrm{Cl}(A)$.
\end{definition}



\begin{theorem}[Charakteryzacja zbioru domkniętego w przestrzeni metrycznej]
Podzbiór $A$ przestrzeni metrycznej $(X, \rho)$ jest domknięty wtedy i tylko wtedy, gdy dla dowolnego ciągu zbieżnego $(x_n)_{n=1}^{\infty},$ zawartego w  $A,$ jego granica również należy do $A.$
\end{theorem}



\begin{definition}[Zbiór gęsty]
Dla danej przestrzeni metrycznej $(X, \rho)$. Zbiór $A \subset X$ nazwiemy gęstym, gdy $\forall_{x \in X} \, \forall_{\epsilon>0} \, \exists_{a \in A} : \rho(x, a) < \epsilon$.
\end{definition}

\begin{definition}[Zbiór brzegowy]
Zbiór $A$ nazwiemy brzegowym, gdy ma puste wnętrze.
\end{definition}

\begin{definition}[Zbiór nigdziegęsty]
Dla danej przestrzeni metrycznej $(X, \rho)$. Zbiór $A \subset X$ nazwiemy nigdziegęstym, gdy wnętrze domknięcia tego zbioru jest puste.
\end{definition}


\begin{definition}[Ciąg]
Ciąg $(x_n)_{n=1}^{\infty}$ jest to funkcja określona na zbiorze liczb naturalnych. Wartości tej funkcji dla kolejnych liczb naturalnych nazywamy wyrazami ciągu i oznaczamy: $x_1, x_2, \ldots$. Niekiedy ciąg oznaczamy skrótowo przez $(x_n)$. W pracy rozważać będziemy ciągi w przestrzeniach metrycznych, czyli funkcje postaci $x: \mathbb{N} \rightarrow X,$ gdzie $X$ jest rozważaną przestrzenią metryczną.
\end{definition}


\begin{definition}[Podciąg]
Niech $(x_n)_{n=1}^{\infty}$ będzie ciągiem o wyrazach ze zbioru $X$. Niech $(k_i)_{i=1}^{\infty} \subset \mathbb{N}$ będzie silnie rosnącym ciągiem indeksów. Wówczas ciąg $(x_{k_i})_{i=1}^{\infty}$ nazywamy podciągiem ciągu $(x_n)_{n=1}^{\infty}.$
\end{definition}

\begin{definition}[Granica ciągu]
W przestrzeni metrycznej $(X, \rho),$ punkt $g$ nazywamy granicą ciągu $(x_n)_{n=1}^\infty$ gdy zachodzi warunek:

$$\forall_{\epsilon>0} \, \exists_{N \in \mathbb{N}} \, \forall_{n>N} \, \rho(x_n, g) < \epsilon.$$
\end{definition}


\begin{definition}[Granica dolna ciągu]
Granicę dolną ciągu $(x_n)_{n=1}^\infty$ definiujemy następująco:

$$\liminf_{n \rightarrow \infty} x_n = \lim_{n \rightarrow \infty} \left(\inf_{k \geq n} x_k\right) = \sup_{n \geq 0} \inf_{k \geq n} x_k.$$
\end{definition}

\begin{definition}[Granica górna ciągu]
Granicę górną ciągu $(x_n)_{n=1}^\infty$ definiujemy następująco:

$$\limsup_{n \rightarrow \infty} x_n = \lim_{n \rightarrow \infty} \left(\sup_{k \geq n} x_k\right) = \inf_{n \geq 0} \sup_{k \geq n} x_k.$$
\end{definition}

\begin{definition}[Punkt skupienia ciągu]
Punkt $x_0$ nazywamy punktem skupienia ciągu $(x_n)_{n=1}^{\infty}$ gdy w dowolnym otoczeniu punktu $x_0$ znajduje się nieskończenie wiele wyrazów tego ciągu.
\end{definition}


\begin{definition}[Ciąg Cauchy'ego]
Ciągiem Cauchy'ego w przestrzeni metrycznej $(X, \rho)$ nazywamy ciąg $(x_n)$ spełniający warunek:
$$\forall_{\epsilon>0} \, \exists_{N \in \mathbb{N}} \, \forall_{n,m > N} \, \rho(x_n, x_m) < \epsilon.$$
\end{definition}



\begin{definition}[Pokrycie otwarte]
\cite[s.~195]{ruette2017chaos}
Pokryciem otwartym przestrzeni metrycznej $X$ nazywamy rodzinę zbiorów otwartych $(U_i)_{i \in I}$ taką, że $X=\bigcup_{i \in I}U_i$.
\end{definition}

\begin{definition}[Przestrzeń metryczna zwarta, definicja pokryciowa]
Na podstawie \cite[s.~196]{ruette2017chaos}.
Przestrzeń metryczną $(X, \rho)$ nazywamy zwartą jeżeli każde pokrycie otwarte $(U_i)_{i \in I}$ tej przestrzeni zawiera podpokrycie skończone, to znaczy istnieje skończony zbiór indeksów $J \subset I,$ taki że $X = \bigcup_{i \in J}U_i$.
Podzbiór $Y \subset X$ jest zwarty jeżeli $(Y, \rho)$ jest przestrzenią zwartą.
\end{definition}

\begin{theorem}[Ciągowa charakteryzacja zwartości przestrzeni metrycznej]
\label{rownowaznosc_definicji_pokryciowej_i_ciagowej}
Przestrzeń metryczna $(X, \rho)$ jest zwarta wtedy i tylko wtedy, gdy z każdego ciągu $(x_n) \subset X$ można wybrać podciąg zbieżny.
\end{theorem}

\begin{proof}
\textcolor{red}{TODO: przyjzec sie temu dowodowi czy pasuje teraz do sformulowania twierdzenia oraz sprobowac poprawic dowod  implikacji w lewa strone (problem z tym ze rodzina tylko przeliczalna a nie uwzgledniamy nieprzeliczalnych)}
$(\Rightarrow)$
Załóżmy, że z dowolnego pokrycia $X$ zbiorami otwartymi możemy wybrać podpokrycie skończone. Weźmy dowolny ciąg $(x_n) \in X$. Pokażę, że ten ciąg ma punkt skupienia w $X$. Załóżmy, że ciąg $(x_n)$ nie ma punktu skupienia w $X$, to znaczy $\forall_{x \in X}$ $x$ nie jest punktem skupienia $x_n$, czyli $\forall_{x \in X} \, \exists_{\epsilon > 0}  K(x, \epsilon) \setminus \{x\} \cap \{x_n\}_{n=1}^{\infty} = \emptyset$. Utwórzmy pokrycie $X$ w następujący sposób: $$\mathcal{A} = \bigcup_{x \in X} K(x, \epsilon_x),$$ gdzie $\epsilon_x > 0$ jest liczbą spełniającą warunek $K(x, \epsilon_x) \setminus \{x\} \cap \{x_n\}_{n=1}^{\infty} = \emptyset$. Oczywiście rodzina $\mathcal{A}$ jest pokryciem otwartym przestrzeni $X$. Zgodnie z założeniem, z pokrycia $\mathcal{A}$ możemy wybrać podpokrycie skończone. Każdy zbiór $A \in \mathcal{A}$ zawiera najwyżej jeden element ciągu $(x_n)$. Skończenie wiele zbiorów zawierających po co najwyżej jednym elemencie ciągu prowadzi do wniosku, że przestrzeń $X$ zawiera skończenie wiele elementów nieskończonego ciągu co stanowi sprzeczność.

Oznaczmy przez $g$ punkt skupienia ciągu $(x_n)$. Oznaczmy przez $y_1$ pierwszy element ciągu $(x_n)$ różny od $g$. przez $y_2$ pierwszy element ciągu $(x_n)$ różny od $g$ i taki, że $y_2 \in K(g, \frac{d}{2})$, gdzie $d = \rho(y_1, g)$.
Następnie niech $y_n$ będzie pierwszym elementem ciągu $(x_n)$ różnym od $g$ i spełniającym warunek $y_n \in K(g, \frac{d_{n-1}}{2})$, gdzie $d_{n-1} = \rho(y_{n-1}, g)$. Powstały w ten sposób podciąg $(y_n) \subset (x_n)$ jest zbieżny do $g \in X$. W przypadku gdyby na którymś etapie powyższej konstrukcji nie można było wskazać kolejnego elementu różnego od $g$, znaczyłoby to, że od pewnego miejsca ciąg $(x_n)$ jest stale równy $g$, a więc zbieżny.

{\color{red}TODO  To trzeba bedzie jeszcze poprawic
$(\Leftarrow)$
Załóżmy, że z każdego ciągu $(x_n) \subset X$ możemy wybrać podciąg zbieżny oraz że istnieje pokrycie $X,$ z którego nie można wybrać podpokrycia skończonego.

Niech $\{\mathcal{O}_i\}_{i \in I}$ będzie pokryciem $X,$ które nie ma podpokrycia skończonego. Na mocy lematu \ref{z_kazdego_pokrycia_mozna_wybrac_podpokrycie_przeliczalne} możemy wybrać z niego podpokrycie przeliczalne.
Załóżmy zatem, że nieskończona, przeliczalna rodzina zbiorów otwartych $\{U_i\}_{i=1}^{\infty}$ stanowi pokrycie $X$. Skonstruujmy następujący ciąg: 
$$x_1 \notin U_1$$
$$x_n \notin \bigcup_{k=1}^{n-1} U_k$$ Zawsze znajdziemy taki $x_n \in X$, gdyż gdyby taki nie istniał znaczyłoby to, że $\bigcup_{i=1}^{n-1}{U_{k_i}} = X$, czyli istniałoby podpokrycie skończone. Załóżmy teraz, że ciąg $(x_n)$ ma podciąg zbieżny. Wtedy istniałaby jego granica, tj. element $g \in X,$ taki że dla każdego $\epsilon > 0$ nieskończenie wiele elementów ciągu leży wewnątrz kuli $B(g, \epsilon)$, z kolei z otwartości zbiorów $U_n$ możemy wybrać taki $\epsilon$, że $B(g, \epsilon) \subset U_k$, gdzie $U_k$ jest dowolnym zbiorem z pokrycia X, zawierającym g. Z konstrukcji naszego ciągu wynika jednak, że $\forall_{n \in N} \forall_{m>n} \, x_m \notin U_n$, czyli do każdego $U_n$ należy jedynie skończenie wiele elementów. Zatem otrzymujemy sprzeczność.
}
\end{proof}

{\color{red} TODO: Dokonczyc
\begin{lemma}
\label{z_kazdego_pokrycia_mozna_wybrac_podpokrycie_przeliczalne}
Niech $(X, \rho)$ będzie przestrzenią metryczną. Jeżeli z każdego ciągu $(x_n) \subset X$ można wybrać podciąg zbieżny, to z każdego pokrycia $X$ można wybrać podpokrycie przeliczalne.
\end{lemma}

\begin{proof}
Najpierw pokażemy, że przestrzeń $X$ jest całkowicie ograniczona, to znaczy:  dla każdego $\epsilon$ można ją pokryć skończenie wieloma kulami o promieniu epsilon;

Ustalmy dowolne $\epsilon > 0.$ Załóżmy, że nie możemy pokryć $X$ skończenie wieloma kulami o promieniu $\epsilon.$ Zatem możemy skonstruować nieskończony ciąg 




2. Wywnioskować, że jest ośrodkowa
3. Posiłkując się kulami o środkach w punktach z ośrodka i promieniach wymiernych, wybrać z dowolnego pokrycia otwartego podpokrycie przeliczalne.
\end{proof}
}


\subsection{Fakty ogólne, dotyczące zwartych przestrzeni metrycznych}


\begin{definition}[Ośrodkowość]
Przestrzeń metryczna $(X, \rho)$ jest ośrodkowa, gdy zawiera podzbiór przeliczalny i gęsty.
\end{definition}


\begin{definition}[Zupełność]
Przestrzeń metryczna jest zupełna jeśli każdy ciąg Cauchy'ego elementów tej przestrzeni jest zbieżny do elementu tej przestrzeni.
\end{definition}


\begin{theorem}[Ośrodkowość zwartych przestrzeni metrycznych] \label{zwarta_jest_osrodkowa}
Jeżeli przestrzeń metryczna jest zwarta to jest również ośrodkowa.
\end{theorem}

\begin{proof}
Niech $(X, \rho)$ będzie przestrzenią metryczną zwartą. Oznacza to, że z każdego pokrycia $X$ zbiorami otwartymi, można wybrać podpokrycie skończone.

Dla $i \in \mathbb{N}$, niech $\mathcal{A}_{i} = \{ K(x, \frac{1}{i}) : x \in X \}$, $\mathcal{A}_{i}$ jest oczywiście pokryciem $X$ zbiorami otwartymi, zatem z każdej rodziny $\mathcal{A}_{i}$ możemy wybrać podpokrycie skończone, oznaczmy je przez $\mathcal{B}_{i}$. Rozważmy teraz następującą rodzinę: 
$$B = \bigcup\limits_{i=1}^{\infty} \mathcal{B}_{i}.$$
Składa się ona ze zbiorów postaci $K(x, \frac{1}{i})$ dla pewnych $i \in \mathbb{N}$ i $x \in X$ i jako przeliczalna suma rodzin skończonych jest przeliczalną rodziną zbiorów.
Niech teraz 
$$C = \{ s : \exists_{i \in \mathbb{N}} \, K\left(s, \frac{1}{i}\right) \in B \}.$$
Zbiór $C$ jest równoliczny ze zbiorem $B$, a więc przeliczalny.
Ponadto $C$ jest gęsty w $X$:
Niech $\epsilon > 0$. Weźmy dowolny $x \in X$. Oczywiście istnieje $i_0 \in \mathbb{N},$ takie że $\frac{1}{i_0} < \epsilon$. Wiemy, że $\mathcal{B}_{i_0}$ stanowi pokrycie $X$ i składa się ze zbiorów postaci $K(s, \frac{1}{i_0})$, gdzie $s \in X$. Skoro $\mathcal{B}_{i_0}$ jest pokryciem $X$ to $\exists_{K(s, \frac{1}{i_0}) \in \mathcal{B}_{i_0}} : x \in K(s, \frac{1}{i_0})$, czyli $\rho(x, s) < \frac{1}{i_0} < \epsilon$, natomiast $s \in C$ z definicji zbioru $C$

Czyli X zawiera podzbiór przeliczalny i gęsty, zatem przestrzeń $(X, \rho)$ jest ośrodkowa.
\end{proof}


\begin{theorem}[Zupełność zwartych przestrzeni metrycznych] \label{zwarta_jest_zupelna}
Każda zwarta przestrzeń metryczna jest zupełna.
\end{theorem}

\begin{proof}
Niech $(X, \rho)$  będzie przestrzenią zwartą.
Weźmy ciąg Cauchy'ego $(x_n)$ elementów $X$. Ze zwartości (twierdzenie \ref{rownowaznosc_definicji_pokryciowej_i_ciagowej}) wynika, że istnieje podciąg $(y_{n_k})$ ciągu $(x_n)$ zbieżny do jakiegoś $g \in X$. Zatem $\forall_{\epsilon > 0} \, \exists_{n_0 \in \mathbb{N}} \, \forall_{n > n_0} \, \rho(y_n, g) < \epsilon$.
Pokażemy, że $(x_n)$ zbiega do $g$.

Ustalmy $\epsilon > 0$. Z faktu, że $(x_n)$ jest ciągiem Cauchy'ego wynika, że $$\exists_{N \in \mathbb{N}} \, \forall_{n, m > N} \, \rho(x_m, x_n) < \frac{\epsilon}{2}$$ Ponadto $$\exists_{n_0 \in \mathbb{N}} \, \forall_{n > n_0} \, \rho(y_n, g) < \frac{\epsilon}{2}$$ Weźmy teraz element $y_{k}$ podciągu $(y_n)$, taki że $y_{k} = x_M$, gdzie $M > N \land M > n_0$. Wówczas 
$$M > N \implies \forall_{n > N} \, \rho(x_M, x_n) < \frac{\epsilon}{2}$$
$$M > n_0 \implies \rho(x_M, g) < \frac{\epsilon}{2}$$

Zatem $\forall_{n > N}: \rho(x_n, g) \leq \rho(x_n, x_M) + \rho(x_M, g) < \frac{\epsilon}{2} + \frac{\epsilon}{2} = \epsilon$, czyli $g \in X$ jest granicą ciągu $(x_n)$, a więc $(x_n)$ jest zbieżny w $(X, \rho)$.
\end{proof}



\subsection{Twierdzenie Baire'a}

\begin{definition}[Zbiór mizerny (zbiór pierwszej kategorii)]
W zupełnej przestrzeni metrycznej zbiorem pierwszej kategorii nazywamy zbiór będący przeliczalną sumą zbiorów nigdziegęstych.
\end{definition}

\begin{definition}[Zbiór rezydualny (zbiór drugiej kategorii)]
\label{definicja_zbioru_rezydualnego}
Zbiór nazywamy rezydualnym, jeżeli zawiera przekrój przeliczalnie wielu gęstych zbiorów otwartych.
\end{definition}



\begin{theorem}[Twierdzenie Baire'a]
\label{twierdzenie_bairea}
W niepustej, zupełnej przestrzeni metrycznej $X$, przekrój przeliczalnie wielu gęstych zbiorów otwartych jest zbiorem gęstym.
\end{theorem}


\begin{lemma}[Przekrój zbiorów rezydualnych jest rezydualny] \label{przekroj_rezydualnych_jest_rezydualny}
W przestrzeni metrycznej zupełnej, przekrój przeliczalnie wielu zbiorów rezydualnych jest rezydualny. W szczególności niepusty.
\end{lemma}



\subsection{Definicje dotyczące układów dynamicznych}

\begin{definition}[Układ dynamiczny]
Układem dynamicznym nazywamy parę $(X, T)$, gdzie $X$ jest zbiorem a $T:X \rightarrow X$ przekształceniem (odwzorowaniem).
\end{definition}

\begin{definition}[Iterata przekształcenia]
Iterata przekształcenia $T$ to potęga przekształcenia $T^n$, $n \in \mathbb{N}$. Potęgę rozumiemy tutaj jako $n-$krotne złożenie odwzorowania z samym sobą: $T^n = (T \circ T \circ \ldots \circ T).$
\end{definition}

\begin{definition}[Orbita]
Orbita punktu $x$ to ciąg $x, Tx, T^2x, T^3x,\ldots$, czyli 
$$O_T(x) = \{T^nx: n=0,1,2,\ldots\}.$$

Jeśli $T$ jest odwracalne (czyli różnowartościowe i ,,na''), to możemy rozważać orbitę obustronną:
$$\{T^nx: n \in \mathbb{Z}\}.$$
\end{definition}

\begin{definition}[Punkt stały]
Punkt $x$ jest stały, gdy $Tx = x$.
\end{definition}

\begin{definition}[Punkt okresowy odwzorowania]
Punkt $x$ jest okresowy, gdy jest stały dla jakiejś iteraty, tzn. gdy $\exists_{n \in \mathbb{N}} \, T^nx = x$.
\end{definition}


\begin{definition}[Orbita okresowa]
Orbitą okresową nazywamy orbitę punktu okresowego. Zauważmy, że orbity okresowe traktowane jako zbiory są zawsze skończone.
\end{definition}

\begin{definition}[Niezmienniczość zbioru ze względu na odwzorowanie]
Zbiór $A \subset X$ nazywamy niezmienniczym ze względu na odwzorowanie $f: X \rightarrow X$ jeżeli $f(A) \subset A.$
\end{definition}



\subsection{Entropia topologiczna}

\subsubsection{Definicja Adlera, Konheima i McAndrew}

\begin{definition}[Entropia topologiczna pokrycia]
Na podstawie \cite{misiurewicz1993}.
Rozważamy układ dynamiczny $(X, f)$, gdzie $X$ jest zwartą przestrzenią metryczną. Niech $\mathcal{A}$ będzie otwartym pokryciem $X$.
Wprowadzamy oznaczenia:

$$\bigvee_{i=1}^n \mathcal{A}_i = \left\{A_1 \cap A_2 \cap \ldots \cap A_n : A_1 \in \mathcal{A}_1, A_2 \in \mathcal{A}_2, \ldots, A_n \in \mathcal{A}_n, A_1 \cap A_2 \cap \ldots \cap A_n \neq \emptyset \right\},$$ 
gdzie $\mathcal{A}_1, \mathcal{A}_2, \ldots, \mathcal{A}_n$ są otwartymi pokryciami $X$.

$$f^{-n}(\mathcal{A}) = \left\{f^{-n}(A):A\in \mathcal{A}\right\},$$

$$\mathcal{A}^n = \bigvee_{i=0}^{n-1}f^{-i}(\mathcal{A}).$$

Ponadto niech $\mathcal{N}(\mathcal{A})$ będzie najmniejszą możliwą mocą podpokrycia wybranego z $\mathcal{A}$ (tzn. licznością podzrodziny rodziny $\mathcal{A}$, która również jest pokryciem $X$). Moc tej rodziny możemy utożsamiać z licznością ponieważ zawsze istnieje podpokrycie skończone, gdyż $X$ jest zwarta.

Wtedy entropią topologiczną odwzorowania $f$ przy pokryciu $\mathcal{A}$ nazywamy wartość:
$$h(f, \mathcal{A}) = \lim_{n \rightarrow \infty}{\frac{1}{n}\log\mathcal{N}(\mathcal{A}^n)}$$
\end{definition}

\begin{definition}[Entropia topologiczna Adler, Konheim, McAndrew]
\label{definicja_entropii_topologicznej_adlera}
Na podstawie \cite{misiurewicz1993}.
Adler, Konheim i McAndrew entropię topologiczną odwzorowania $f$ definiują następująco: 
$$h(f) = \sup h(f, \mathcal{A}),$$ gdzie supremum jest brane po wszystkich otwartych pokryciach $\mathcal{A}$ przestrzeni $X$.
\end{definition}


\subsubsection{Definicja Bowena}

Niech $(X, \rho)$ będzie zwartą przestrzenią metryczną, a $f: X \rightarrow X$ odwzorowaniem ciągłym. Rozważamy układ dynamiczny $(X, f).$ {\color{red}(TODO: ma byc uniformly continous - tak jest u Bowena, jest tak ze zwartości $X$, napisać to tutaj?)}.

\begin{definition}[Kula Bowena]
\cite[s.~58]{ruette2017chaos}
Niech  $\epsilon > 0$ oraz $n \geq 1$. Kulę Bowena rzędu $n$ o środku w punkcie $x \in X$ i promieniu $\epsilon$ definiujemy następująco

\begin{equation}
\begin{aligned}
B_n(x,\epsilon) & \coloneqq \left\{y \in X : \rho\left(f^k(x), f^k(y)\right) \leq \epsilon, k \in \{0, 1, \ldots, n-1\}\right\} \\
& = \bigcap_{i=0}^{n-1}f^{-i}\left(\overline{K}\big(f^i(x), \epsilon\big)\right).
\end{aligned}
\end{equation}

\end{definition} 


\begin{definition}[Zbiór $(n,\epsilon)$-rozdzielony]
\label{definicja_n_eps_rozdzielonego}
\cite[s.~58]{ruette2017chaos}
Mówimy, że zbiór $E \subset X$ jest $(n, \epsilon)$-rozdzielony, jeżeli dla każdych $x, y \in E,$ takich że $x \neq y$ istnieje $k \in \{0, 1, \ldots, n-1\},$ takie że $\rho\left(f^k(x), f^k(y)\right) > \epsilon.$

Maksymalną moc zbioru $(n, \epsilon)$-rozdzielonego oznaczamy przez $s_n(f, \epsilon).$
\end{definition}


\begin{definition}[Zbiór $(n, \epsilon)$-rozpinający]
\cite[s.~58]{ruette2017chaos}
Zbiór $E$ nazywamy $(n,\epsilon)$-rozpinającym jeżeli $X \subset \bigcup_{x \in E}B_n(x, \epsilon)$.

Minimalną moc zbioru $(n, \epsilon)$-rozpinającego oznaczamy przez $r_n(f, \epsilon).$

\end{definition}


\begin{definition}[Entropia topologiczna Bowena]
\label{definicja_entropii_topologicznej_bowena}
\cite[s.~59]{ruette2017chaos}
Entropię topologiczną $h_{\textrm{top}}(f)$ odwzorowania $f$ definiujemy następująco:

$$h_{\textrm{top}}(f) = \lim_{\epsilon \rightarrow 0} \limsup_{n \rightarrow +\infty} \frac{1}{n} \log s_n(f, \epsilon) = \lim_{\epsilon \rightarrow 0} \limsup_{n \rightarrow +\infty} \frac{1}{n} \log r_n(f, \epsilon).$$

\end{definition}


\begin{definition}[Entropia topologiczna Bowena na podzbiorze przestrzeni]
\label{entropia_bowena_podzbior_przestrzeni}
\cite[s.~402]{bowen1971entropy__do_dowodu_rownosci_topologicznej_entropii_2_twierdzenie17}
Powyższe definicje występują też w wersjach ograniczonych do ustalonego zbioru zwartego $K \subset X.$ I tak, niech $K \subset X$ będzie zbiorem zwartym, wtedy:
\begin{enumerate}
\item Maksymalną moc zbioru $(n, \epsilon)$-rozdzielonego zawartego w $K$ oznaczamy przez $s_n(f, \epsilon, K).$
\item $h_{\textrm{top}}(f, K) = \lim_{\epsilon \rightarrow 0} \limsup_{n \rightarrow +\infty} \frac{1}{n} \log s_n(f, \epsilon, K).$
\end{enumerate}

\end{definition}


\begin{theorem}
\cite[s.~402]{bowen1971entropy__do_dowodu_rownosci_topologicznej_entropii_2_twierdzenie17}
Zachodzi następująca równość:
$$h_{\textrm{top}}(f) = \sup_{K \subset X, K \textrm{ zwarty}}h_{\textrm{top}}(f, K)$$
\end{theorem}

W dalszej części pracy będziemy używali oznaczenia $h$ zamiast $h_{\textrm{top}},$ w miejscach gdzie nie będzie prowadziło to do nieporozumień.



\begin{theorem}(Równość entropii Adlera, Konheima, McAndrew i entropii topologicznej Bowena)
Definicje \ref{definicja_entropii_topologicznej_adlera} i \ref{definicja_entropii_topologicznej_bowena} entropii topologicznej są równoważne. \cite[s.~59]{ruette2017chaos}
\end{theorem}

 

\begin{lemma}[Złożenie dwóch funkcji niemalejących jest niemalejące]
\label{zlozenie_dwoch_niemalejacych_jest_niemalejace}
Niech $f: [0, 1] \rightarrow [0, 1]$ i $g: [0, 1] \rightarrow [0,1]$ będą funkcjami niemalejącymi, wówczas ich złożenie $f \circ g$ również jest funkcją niemalejącą. 
\end{lemma}

\begin{proof}
Weźmy dowolne $x, y \in [0,1]$, niech $y < x$. Wówczas, ponieważ $g$ jest niemalejąca zachodzi nierówność $g(y) \leq g(x)$, z faktu, że $f$ również jest niemalejąca otrzymujemy $f(g(y)) \leq f(g(x)).$ Czyli $(f \circ g)(y) \leq (f \circ g)(x).$
\end{proof} 

\begin{corollary}[Złożenie skończenie wielu funkcji niemalejących jest niemalejące]
\label{zlozenie_skonczenie_wielu_niemalejacych_jest_niemalejace}
Wniosek ten otrzymujemy z lematu \ref{zlozenie_dwoch_niemalejacych_jest_niemalejace} przez zastosowanie indukcji matematycznej.
\end{corollary}

\begin{corollary}[Iteraty funkcji niemalejącej są niemalejące]
\label{iteraty_niemalejacych_sa_niemalejace}
Niech $f: [0, 1] \rightarrow [0, 1]$ będzie funkcją niemalejącą, wówczas dla każdego $i \in \mathbb{N}$ funkcje $f^{i}$ są niemalejące.
\end{corollary}

 
 
\begin{theorem}[Entropia topologiczna odwzorowań monotonicznych]
\label{entropia_topologiczna_odwzorowan_monotonicznych}

Niech $I = [0, 1]$ i niech $f: I \rightarrow I$ będzie funkcją monotoniczną (niekoniecznie ściśle).
Wówczas entropia topologiczna $f$ jest równa zero.
\end{theorem}

\begin{proof}
Rozważmy dwa przypadki: osobno dla $f: I \rightarrow I$ niemalejącej i nierosnącej.
\item[Przypadek 1: $f: I \rightarrow I$ niemalejąca.]

Skorzystamy z definicji entropii topologicznej Bowena. Pokażemy, że 
$$\lim_{\epsilon \rightarrow 0} \limsup_{n \rightarrow +\infty} \frac{1}{n} \log s_n(f, \epsilon) = 0.$$

% Pokażemy, że $\forall_{n \in \mathbb{N}, \epsilon>0} \, s_n(f, \epsilon) \leq \frac{n}{\epsilon}.$

Ustalmy dowolne $n \in \mathbb{N}$ i $\epsilon > 0.$ Podzielmy odcinek $[0, 1]$ na $m = \lfloor 1/\epsilon \rfloor + 1 \leq \frac{1}{\epsilon} + 1$ odcinków o (niekoniecznie jednakowych) długościach równych $d_{i,\epsilon} < \epsilon$ w taki sposób aby otrzymać rodzinę odcinków: 
$$\left\{ [0, d_{1,\epsilon}), [d_{1,\epsilon}, d_{1,\epsilon} + d_{2, \epsilon}), \ldots, [\sum_{i=1}^{m-1} d_{i, \epsilon}, 1] \right\}.$$
Oznaczmy krańce tych odcinków przez $a_0=0, \, a_1 = d_{1,\epsilon}, \, a_2 = d_{1,\epsilon} + d_{2,\epsilon}, \, \ldots, \, a_m = 1.$ 

Zdefiniujmy $A_i \coloneqq \{f^{-i}(a_0), f^{-i}(a_1), \ldots, f^{-i}(a_m)\}$, dla $i \in \{0, 1, \ldots, n-1\}$

Niech $B = \bigcup_{i=0}^{n-1} A_i.$ Każdy ze zbiorów $A_i$ zawiera $m+1$ elementów, zatem $|B| \leq n \cdot (m+1)$ Uporządkujmy elementy zbioru $B$ w niemalejący ciąg $(b_j)_{j=0}^{|B|-1}.$ Podzielmy $I$ na odcinki postaci $B_j = [b_{j-1}, b_{j}]$ dla $j \in \{1, 2, \ldots, |B|-1\}.$  

Pokażę, że dowolne dwa punkty $x, y$ należące do tego samego odcinka $B_j$ nie mogą jednocześnie należeć do tego samego zbioru $(n, \epsilon)$-rozdzielonego.

Weźmy dowolny odcinek $B_j = [b_{j-1}, b_{j}] \, j \in \{1, 2, \ldots, |B|-1\}.$ Wybierzmy z niego dwa dowolne punkty $x, y$, $x \neq y$. Bez straty ogólności, niech $x > y.$ Oczywiście nie istnieje punkt $b_k \in B$ dla którego $y < b_k < x.$ Z konstrukcji $B$ widzimy natychmiast, że $\forall_{i \in \{0, 1, \ldots, n-1\}} \, \exists_{j \in \{1,2,\ldots,m\}} \, f^{-i}(a_{j-1}) \leq y < x \leq f^{-i}(a_j)$. Z monotoniczności funkcji $f$ i~wniosku \ref{iteraty_niemalejacych_sa_niemalejace} otrzymujemy 
$$f^i(f^{-i}(a_{j-1})) =  a_{j-1} \leq f^i(y) \leq f^i(x) \leq a_j = f^i(f^{-i}(a_j)).$$
Ponieważ $\rho(a_{j-1}, a_j) = d_{j,\epsilon} < \epsilon$ to  $\rho(f^i(x), f^i(y)) < \epsilon.$ A więc $x$ i $y$ nie należą do jednego zbioru $(n, \epsilon)$-rozdzielonego.



Przypomnijmy, że $s_n(f, \epsilon)$ oznacza maksymalną moc zbioru $(n, \epsilon)$-rozdzielonego. 

% Aby dwa punkty należały do zbioru $(n, \epsilon)$-rozdzielonego, nie mogą one należeć do jednej kuli Bowena o promieniu mniejszym bądź równym $\frac{\epsilon}{2}.$ 

Z wcześniejszych rozważań wynika, że do dowolnego zbioru $(n, \epsilon)$-rozdzielonego może należeć co najwyżej jeden punkt z każdego z odcinków $B_j$, odcinków tych jest $|B|-1.$ Zatem $s_n(f, \epsilon) \leq |B|-1 \leq n \cdot (m+1) - 1 \leq n \cdot (\frac{1}{\epsilon}+1+1) - 1 = \frac{n}{\epsilon} + \frac{n}{2} - 1.$



Ostatecznie otrzymujemy:
\begin{equation}
\begin{aligned}
h_{\textrm{top}}(f) & = \lim_{\epsilon \rightarrow 0} \limsup_{n \rightarrow +\infty} \frac{1}{n} \log s_n(f, \epsilon) \leq \lim_{\epsilon \rightarrow 0} \limsup_{n \rightarrow +\infty} \frac{1}{n} \log \big( n \cdot (m+1) - 1 \big) \\
& \leq \lim_{\epsilon \rightarrow 0} \limsup_{n \rightarrow +\infty} \frac{1}{n} \log \left( \frac{n}{\epsilon} + \frac{n}{2} - 1 \right) \leq \lim_{\epsilon \rightarrow 0} \limsup_{n \rightarrow +\infty} \frac{1}{n} \log \left( \frac{2 \cdot n + \epsilon \cdot n}{2 \cdot \epsilon}\right) \\
& = \lim_{\epsilon \rightarrow 0} \limsup_{n \rightarrow +\infty} \left(\frac{1}{n} \log \big(n \cdot (2 + \epsilon)\big) - \frac{1}{n} \log ( 2 \cdot \epsilon )\right) = \lim_{\epsilon \rightarrow 0} 0 = 0
\end{aligned}
\end{equation}

Entropia z definicji jest nieujemna, więc $h_{\textrm{top}}(f) = 0.$

\item[Przypadek 2: $f: I \rightarrow I$ nierosnąca.]

Jeżeli $f$ jest nierosnąca, to $f^2$ jest niemalejąca. Zatem z przypadku pierwszego $h_{\textrm{top}}(f^2) = 0$. Dla entropii topologicznej zachodzi następująca równość: 
$$h_{\textrm{top}}(f^n) = n \cdot h_{\textrm{top}}(f),$$
zatem 
$0 = h_{\textrm{top}}(f^2) = 2 \cdot h_{\textrm{top}}(f),$
czyli
$h_{\textrm{top}}(f) = 0.$

\end{proof}
 
 


\subsection{Chaos}


Pomimo dużej popularności chaotycznych układów dynamicznych, przez długi czas nie było jednej powszechnie akceptowanej definicji chaosu. W niniejszej pracy prezentujemy dwie spośród powszechnie stosowanych: chaos w sensie Devaneya oraz w sensie Li-Yorke'a. W dalszej części pracy korzystać zajmować się będziemy odwzorowaniami chaotycznymi w sensie Devaneya.

\begin{definition}[Topologiczna tranzytywność]
\cite{onDeveneyDefinitionOfChaos}
Mówimy, że $f$ jest topologicznie tranzytywne w przestrzeni metrycznej $X$, gdy dla każdych niepustych, otwartych podzbiorów $U, V$ przestrzeni $X$ istnieje liczba naturalna $k$ taka, że przekrój $f^k(U) \cap V$ jest niepusty.
\end{definition}

\begin{definition}[Wrażliwość na warunki początkowe]
\cite{balibrea2003topological}
Niech $(X, \rho)$ będzie przestrzenią metryczną, a $f: X \rightarrow X$ odwzorowaniem ciągłym.
Mówimy, że $f$ jest wrażliwe na warunki początkowe jeżeli  istnieje $\delta > 0,$ taki że dla każdego $x \in X$ i dla każdego otoczenia $U$ punktu $x$, istnieją $y \in U$ i $n \geq 0$ dla których zachodzi $\rho(f^n(x), f^n(y)) > \delta$.
\end{definition}

\begin{definition}[Chaos w sensie Devaneya]
Niech $X$ będzie przestrzenią metryczną. Ciągłe odwzorowanie $f: X \rightarrow X$ nazywamy chaotycznym (w sensie Devaneya) na $X$ jeżeli:
\begin{enumerate}
\item $f$ jest tranzytywne,
\item zbiór punktów okresowych $f$ jest gęsty w $X$,
\item $f$ jest wrażliwe na warunki początkowe.
\end{enumerate}
\end{definition}


\begin{theorem}
[Warunki dostateczne chaotyczności w sensie Devaneya]
\label{warunki_dostateczne_chaotycznosci_devaneya}
\cite{onDeveneyDefinitionOfChaos}
Niech $(X, \rho)$ będzie przestrzenią metryczną oraz niech $X$ nie będzie zbiorem skończonym.
Jeżeli $f: X \rightarrow X$ jest tranzytywne i posiada gęsty zbiór punktów okresowych to $f$ jest wrażliwe na warunki początkowe.
\end{theorem}

\begin{proof}
\cite{onDeveneyDefinitionOfChaos}
Załóżmy, że $f: X \rightarrow X$ jest tranzytywne i posiada gęsty zbiór punktów okresowych.

W pierwszej kolejności zauważmy, że istnieje liczba $\delta_0 > 0$ taka, że dla każdego $x \in X$ istnieje punkt okresowy $q \in X$, którego orbita $O(q)$ jest w odległości co najmniej $\frac{\delta_0}{2}$ od $x$: 

Wybierzmy dwa dowolne punkty okresowe $q_1$ i $q_2$ o rozłącznych orbitach $O(q_1), O(q_2)$. 


Takie punkty istnieją, ponieważ:
\begin{enumerate}
\item Orbity okresowe są skończone, więc żadna taka orbita nie może być gęsta w nieskończonym zbiorze $X.$
\item Zbiór punktów okresowych odwzorowania $f$ jest gęsty w $X,$ więc istnieją punkty okresowe, należące do różnych orbit.
\item Rozważmy dwie orbity $O(p), O(q)$, bez straty ogólności niech $|O(p)| \geq |O(q)|$. Jeżeli te orbity są różne to $\exists_{p_0 \in O(p)} \, p_0 \notin O(q).$ Przypuśćmy, że istnieje $n \in \mathbb{N},$ takie że $f^n(p_0) \in O(q).$ Wówczas, ponieważ $O(q)$ jest orbitą okresową, istniałoby $k \in \mathbb{N},$ takie że $O(q) \ni f^{n+k}(p_0) = p_0,$ co stanowi sprzeczność. Zatem $\forall{n \in \mathbb{N}} \, f^n(p_0) \notin O(q),$ czyli jeżeli orbity są różne to są one rozłączne.
\end{enumerate}


Oznaczmy przez $\delta_0$ odległość między $O(q_1)$ a $O(q_2)$. Wówczas z nierówności trójkąta, każdy punkt $x \in X$ jest w odległości co najmniej $\frac{\delta_0}{2}$ od  jednej z dwóch wybranych orbit.
Pokażemy, że $f$ jest wrażliwe na warunki początkowe ze stałą (ang. \textit{sensitivity constant}) $\delta = \frac{\delta_0}{8}.$

Weźmy dowolny punkt $x \in X$ i niech $N$ będzie pewnym otoczeniem punktu $x$. Ponieważ zbiór punktów okresowych odwzorowania $f$ jest gęsty, to istnieje punkt okresowy $p$ należący do przekroju $U = N \cap K(x, \delta)$. Niech $n$ oznacza okres punktu $p$. Z wcześniejszych rozważań wiemy, że istnieje punkt okresowy $q \in X$, którego orbita $O(q)$ jest w odległości co najmniej $4\delta$ od $x$. Oznaczmy

$$V = \bigcap_{i=0}^n f^{-i}(K(f^i(q), \delta)).$$

Zbiór $V$ jest otwarty i niepusty, gdyż $q \in V.$ W związku z tym, ponieważ $f$ jest tranzytywne, to istnieją $y \in U$ i $k \in \mathbb{N},$ takie że $f^k(y) \in V$.

Niech teraz $j$ będzie częścią całkowitą z $\frac{k}{n} + 1$. Wtedy $1 \leq nj - k \leq n$. Z konstrukcji mamy

$$f^{nj}(y) = f^{nj-k}\left(f^k(y)\right) \in f^{nj-k}(V) \subseteq K\left(f^{nj-k}(q), \delta\right).$$

Liczba $n$ jest okresem $p$, więc $f^{nj}(p) = p$, a z nierówności trójkąta:

$$\rho\left(x, f^{nj-k}(q)\right) \leq \rho(x, p) + \rho\left(p, f^{nj}(y)\right) + \rho\left(f^{nj}(y), f^{nj-k}(q)\right),$$

po przekształceniach dostajemy:

\begin{equation}
\begin{aligned}
\rho\left(f^{nj}(p), f^{nj}(y)\right) & = \rho\left(p, f^{nj}(y)\right) \\
& \geq \rho\left(x, f^{nj-k}(q)\right) - \rho\left(f^{nj-k}(q), f^{nj}(y)\right) - \rho\left(p, x\right),
\end{aligned}
\end{equation}
Następnie, ponieważ $p \in K(x, \delta)$ oraz $f^{nj}(y) \in K\left(f^{nj-k}(q), \delta\right)$ otrzymujemy

$$\rho\left(f^{nj}(p), f^{nj}(y)\right) > 4\delta - \delta - \delta = 2\delta.$$

Ostatecznie, korzystając z nierówności trójkąta dostajemy, że albo $\rho\left(f^{nj}(x), f^{nj}(y)\right) > \delta$, albo $\rho\left(f^{nj}(x), f^{nj}(p)\right) > \delta.$ W obu przypadkach znaleźliśmy punkt, którego $nj$-ta iterata jest w odległości większej od $\delta$ od $f^{nj}(x).$
\end{proof}



\begin{definition}[Chaos w sensie Li-Yorke'a]
\cite[s.~25]{aulbach2001three}
Ciągłe odwzorowanie $f: X \rightarrow X$ na zwartej przestrzeni metrycznej $(X, \rho)$ jest chaotyczne (w sensie Li-Yorke'a) jeżeli istnieje nieprzeliczalny podzbiór $S$ przestrzeni $X$ o następujących własnościach:
\begin{enumerate}
\item $\limsup_{n \rightarrow \infty} \rho\left(f^n(x), f^n(y)\right) > 0$ dla każdych $x,y \in S, \, x \neq y,$
\item $\liminf_{n \rightarrow \infty} \rho\left(f^n(x), f^n(y)\right) = 0$ dla każdych $x,y \in S, \, x \neq y,$
\item $\limsup_{n \rightarrow \infty} \rho\left(f^n(x), f^n(p)\right) > 0$ dla każdych $x \in S, \, p \in X, \, p \textrm{ - okresowy.}$
\end{enumerate}
\end{definition}




\subsection{Tranzytywność a entropia}

Zarówno tranzytywność, chaotyczność jak i topologiczną entropię możemy intuicyjnie pojmować jako pewne ,,miary skomplikowania'' układu dynamicznego. Pojawia się więc pytanie czy są to pojęcia tożsame, a jeżeli nie, to jak bardzo się różnią. Czy możliwe jest aby odwzorowanie topologicznie tranzytywne miało zerową entropię topologiczną i odwrotnie: czy odwzorowanie o niezerowej entropii topologicznej może nie być tranzytywne. Okazuje się, że odpowiedź na oba te pytania jest twierdząca. Wiele zależy od przestrzeni którą rozważamy.
Mając daną przestrzeń, możemy zadać pytanie czy istnieje dla niej odwzorowanie tranzytywne (a nawet więcej: chaotyczne), ale o zerowej entropii topologicznej. Ogólniej: jaką wartość przyjmuje dolne ograniczenie na entropię topologiczną dla odwzorowań chaotycznych na danej przestrzeni i czy istnieje odwzorowanie o entropii równej temu ograniczeniu. Innymi słowy możemy rozważać problem istnienia i wartości minimum bądź supremum entropii topologicznej po wszystkich chaotycznych odwzorowaniach danej przestrzeni w nią samą. Poniżej przedstawiamy kilka znanych wyników przedstawionych w \cite{balibrea2003topological}.
W każdym z poniższych punktów zakładamy, że $f$ jest chaotyczne w sensie Devaneya.
\begin{enumerate}
\item Jeżeli $f: [0,1] \rightarrow [0,1],$ to entropia topologiczna $h(f) \geq \frac{1}{2} \log(2)$.
\item Istnieje $f: C \rightarrow C,$ gdzie $C$ jest zbiorem Cantora, takie że $h(f) = 0.$
\item Niech $F = \{f : S^1 \rightarrow S^1 \, | \, f \textrm{ jest chaotyczne w sensie Devaneya}\},$ gdzie $S^1$ oznacza okrąg. Dla każdego $\epsilon > 0$ istnieje $f \in F$ dla którego $h(f) < \epsilon,$ jednocześnie nie istnieje takie $f \in F,$ dla którego wartość entropii osiąga zero.
\end{enumerate}



\subsection{Odwzorowania trójkątne}


\begin{definition}[Odwzorowanie trójkątne]
Rozważmy ,,prostokąt'' $X \times Y$, gdzie $X, Y$ są przestrzeniami metrycznymi. 
Odwzorowaniem trójkątnym nazywamy funkcję $F: X \times Y \rightarrow X \times Y$ postaci:
$$F(x,y) = \big(f(x), g(x,y)\big).$$
\end{definition}


\begin{definition}[Ciągłość odwzorowań trójkątnych]
W pracy rozważać będziemy wyłącznie odwzorowania ciągłe, tj. takie trójkątne $F$ dla których $f \in C(X)$ i $g$ jest ciągłą funkcją z $X \times Y$ w $Y$. 
Zbiór takich odwzorowań oznaczać będziemy $C_\triangle(X \times Y)$
Zamiast $g(x, y)$ będziemy pisać $g_x(y)$, gdzie $g_x: Y \rightarrow Y$ jest rodziną ciągłych przekształceń zależną w sposób ciągły od $x \in X$.
\end{definition}


\begin{definition}[Odwzorowanie bazowe]
Odwzorowanie $f$ nazywamy odwzorowaniem bazowym (ang. \textit{basis map}) odwzorowania $F = (f, g_x)$
\end{definition}

\begin{definition}[Odwzorowania włóknowe]
Odwzorowania $g_x$ dla $x \in X$ nazywamy odwzorowaniami włóknowymi (ang. \textit{fibre maps}) odwzorowania $F = (f, g_x)$.
\end{definition}


\begin{theorem}[O równości entropii odwzorowania trójkątnego i jego odwzorowania bazowego]
\label{rownosc_entropii_gdy_wlokna_monotoniczne}
Niech $X$ będzie zwartą przestrzenią metryczną, $I = [0, 1]$, $F = (f, g_x) \in C_\triangle(X \times I).$ Wtedy, jeżeli $\forall_{x \in X}$ funkcje włóknowe $g_x$ są (niekoniecznie ściśle) monotoniczne, to entropia topologiczna odwzorowania $F$ jest równa entropii topologicznej jego odwzorowania bazowego $f$, czyli:

$$h(F) = h(f)$$
\end{theorem}

\begin{proof}
%\cite{do_dowodu_rownosci_topologicznej_entropii_2_twierdzenieD}

Rozważamy dwie zwarte przestrzenie metryczne: $(X, d_1)$ oraz $(X \times I, d_2)$ z metrykami zdefiniowanymi w \ref{definicja_metryki_d_jeden} i \ref{definicja_metryki_d_dwa}. Przestrzeń $X$ jest zwarta z założenia, natomiast przestrzeń $X \times I$ jest zwarta jako produkt dwóch przestrzeni zwartych. 

Zdefiniujmy odwzorowanie $\pi: X \times I \rightarrow X$, $\pi(x,a) = x$. Oczywiście $\pi$ jest ciągłe i ,,na''.
Wówczas $F: X \times I \rightarrow X \times Y$, $f: X \rightarrow X$ oraz $\pi: X \times I \rightarrow X$ (,,na'') są ciągłymi odwzorowaniami. 

Weźmy dowolny punkt $(x, a) \in X \times I$:
$$(\pi \circ F)(x, a) = \pi\big(F(x,a)\big) = \pi\big(f(x), g_x(a)\big) = f(x),$$
$$(f \circ \pi)(x, a) = f\big(\pi(x,a)\big) = f(x),$$
a więc $\pi \circ F = f \circ \pi.$


Na podstawie twierdzenia 17 z pracy  \cite[s.~409]{bowen1971entropy__do_dowodu_rownosci_topologicznej_entropii_2_twierdzenie17} otrzymujemy, że 

$$h(F) \leq h(f) + \sup_{x \in X} h(F, \pi^{-1}(x))$$


Ustalmy $x \in X.$ Wiemy, że:
$$h(F, \pi^{-1}(x)) = h(F, \{x\} \times I) = \lim_{\epsilon \rightarrow 0} \limsup_{n \rightarrow +\infty} \frac{1}{n} \log s_n(F, \epsilon, \{x\} \times I),$$

gdzie $s_n(F, \epsilon, \{x\} \times I)$ jest maksymalną mocą zbioru $(n, \epsilon)-$rozdzielonego, zawartego w $\{x\} \times I.$

Zauważmy, że zbiór $\{x\} \times I$ jest $(n, \epsilon)$-rozdzielony, jeżeli dla każdych $(x, y_1), (x, y_2) \in \{x\} \times I,$ takich że $y_1 \neq y_2$ istnieje $k \in \{0, 1, \ldots, n-1\},$ takie że 

\begin{equation}
\label{nierownosc_zbioru_rozdzielonego_d2}
\begin{aligned}
& d_2(F^k(x, y_1), F^k(x, y_2)) \\
& = d_2 \big( \big(f^k(x), (g_{f^{k-1}(x)} \circ \ldots \circ g_{f(x)} \circ g_x)(y_1) \big), \big(f^k(x), (g_{f^{k-1}(x)} \circ \ldots \circ g_{f(x)} \circ g_x)(y_2) \big)
\big) \\
& = d_1\big((g_{f^{k-1}(x)} \circ \ldots \circ g_{f(x)} \circ g_x)(y_1), (g_{f^{k-1}(x)} \circ \ldots \circ g_{f(x)} \circ g_x)(y_2)\big) \\
& > \epsilon.
\end{aligned}
\end{equation}


 Podstawmy nierówność \ref{nierownosc_zbioru_rozdzielonego_d2} w miejsce $\rho(f^k(x), f^k(y)) > \epsilon$ z definicji \ref{definicja_n_eps_rozdzielonego}. Przedefiniujmy zbiór $A_i$  z twierdzenia \ref{entropia_topologiczna_odwzorowan_monotonicznych} w następujący sposób:
 $A_i \coloneqq \{F^{-i}(x, a_0), F^{-i}(x, a_1), \ldots, F^{-i}(x, a_m)\}$, dla $i \in \{0, 1, \ldots, n-1\}.$ 
Dla każdych $i \in \mathbb{N}, \, x \in X, \, a \in I$ zachodzi:
$$F^{-i}(x, a) = \big( f^{-i}(x), (g_{f^{-i}(x)}^{-1}\circ \ldots \circ g_{f^{-2}(x)}^{-1} \circ g_{f^{-1}(x)}^{-1})(a) \big)$$
Wszystkie odwzorowania włóknowe $g_x$ są niemalejące, a więc ich złożenia również (na mocy wniosku \ref{zlozenie_skonczenie_wielu_niemalejacych_jest_niemalejace}).

W takim razie rozumując analogicznie jak w twierdzeniu \ref{entropia_topologiczna_odwzorowan_monotonicznych}, łatwo stwierdzamy, że:

$$\forall_{x \in X} \, h(F, \pi^{-1}(x)) = 0,$$
czyli również:
$$\sup_{x \in X} h(F, \pi^{-1}(x)) = 0.$$



Otrzymujemy zatem nierówność:
\begin{equation}
\label{twierdzenie_o_rownosci_entropii_nierownosc_pierwsza}
h(F) \leq h(f)
\end{equation}


Pokażemy teraz, że $h(F) \geq h(f).$
Skorzystamy z definicji \ref{definicja_entropii_topologicznej_adlera} (Adlera) entropii topologicznej. Rozważmy wszystkie pokrycia otwarte przestrzeni $X \times I$ postaci $\widetilde{\mathcal{A}} = \{A \times I : A \in \mathcal{A}\},$ gdzie $\mathcal{A}$ jest dowolnym pokryciem otwartym przestrzeni $X.$ Wtedy dla każdego takiego pokrycia $\widetilde{\mathcal{A}}$ zachodzi:
\begin{align*}
F^{-n}(A \times I) & = \{(x,y): F^n(x,y) \in A \times I\} \\
& = \left\{(x,y): f^n(x) \in A \wedge \big(g_{f^n(x)} \circ \ldots \circ g_{f(x)} \circ g_x\big)(y) \in I\right\} \\
& = \{x : f^n(x) \in A\} \times I \\
& = f^{-n}(A) \times I,
\end{align*}
dla każdego $A \times I \in \widetilde{\mathcal{A}}.$ 
Dalej 
$$F^{-n}\left(\widetilde{\mathcal{A}}\right) = \left\{F^{-n}(A \times I) : A \times I \in \widetilde{\mathcal{A}}\right\} = \left\{f^{-n}(A) \times I : A \in \mathcal{A}\right\}.$$
Ponadto dla każdych $A_1 \times I, A_2 \times I$ mamy $(A_1 \times I) \cap (A_2 \times I) = (A_1 \cap A_2) \times I,$ a więc $${\widetilde{\mathcal{A}}}^n = \bigvee_{i=0}^{n-1}F^{-i}\left(\widetilde{\mathcal{A}}\right) = \left\{A \times I : A \in \bigvee_{i=0}^{n-1}f^{-i}({\mathcal{A}}) \right\}.$$

Skoro dla każdego $n \in \mathbb{N}$ wszystkie zbiory należące do ${\widetilde{\mathcal{A}}}^n$ są postaci jak wyżej, to podrodzina ${\widetilde{\mathcal{B}}}^n \subseteq {\widetilde{\mathcal{A}}}^n$ jest podpokryciem przestrzeni $X \times I$ wtedy i tylko wtedy, gdy podrodzina $\mathcal{B}^n \subseteq \mathcal{A}^n$ jest podpokryciem przestrzeni $X.$
Z powyższego wynika, że dla każdego $n \in \mathbb{N}$ zachodzi
$\mathcal{N}\left({\widetilde{\mathcal{A}}}^n\right) = \mathcal{N}(\mathcal{A}^n),$
co w konsekwencji daje:
$$\sup_{\widetilde{\mathcal{A}} \in \mathcal{R}} h(F, \widetilde{\mathcal{A}}) = \sup_{\mathcal{A}} h(f, \mathcal{A}),$$
gdzie 
$$\mathcal{R} = \left\{\widetilde{\mathcal{A}} : \widetilde{\mathcal{A}} = \{A \times I : A \in \mathcal{A}\}, \textrm{ gdzie } \mathcal{A} \textrm{ jest dowolnym pokryciem otwartym przestrzeni } X\right\},$$
Drugie supremum w powyższym wzorze brane jest po wszystkich pokryciach przestrzeni $X,$ czyli otrzymujemy:
$$\sup_{\widetilde{\mathcal{A}} \in \mathcal{R}} h(F, \widetilde{\mathcal{A}}) = h(f).$$

Aby otrzymać entropię odwzorowania $F$ bierzemy supremum po wszystkich pokryciach przestrzeni $X \times I,$ czyli po nadzbiorze zbioru $\mathcal{R},$ więc:
\begin{equation}
\label{twierdzenie_o_rownosci_entropii_nierownosc_druga}
h(F) \geq \sup_{\widetilde{\mathcal{A}} \in \mathcal{R}} h(F, \widetilde{\mathcal{A}})  = h(f)
\end{equation}


Ostatecznie z nierówności \ref{twierdzenie_o_rownosci_entropii_nierownosc_pierwsza} i \ref{twierdzenie_o_rownosci_entropii_nierownosc_druga} dostajemy:
$$h(F) = h(f).$$


\end{proof}



\subsection{Definicje utworzone na potrzeby dowodu twierdzenia o rozszerzaniu}
Definicje i własności odwzorowań trójkątnych podajemy na podstawie pracy \cite{balibrea2003topological}.

Na potrzeby dowodu wprowadźmy pojęcia odległości między odwzorowaniami oraz dwie funkcje: $\textrm{pr}_1(x, y)$ i $\textrm{pr}_2(x, y)$. 

\begin{definition}[Metryka na przestrzeni ciągłych przekształceń zwartej przestrzeni metrycznej]
\label{definicja_metryki_d_jeden}
Niech $(M, \sigma)$ będzie zwartą przestrzenią metryczną, rozważmy odwzorowania $h,k \in C(M)$. Odległość między nimi zdefiniujmy jako $\max_{m \in M} \sigma(h(m), k(m))$ i oznaczmy ją jako $d_1(h,k)$.
\end{definition}


\begin{definition}[Metryka na przestrzeni odwzorowań trójkątnych]
\label{definicja_metryki_d_dwa}
Odległość między odwzorowaniami trójkątnymi definiujemy następująco: Niech $(X, \rho)$ i $(Y, \tau)$ będą zwartymi przestrzeniami metrycznymi a $F(x,y) = (f(x), g_x(y))$ i $\Phi(x,y) = (\phi(x), \psi_x(y))$ trójkątnymi odwzorowaniami należącymi do $C_\triangle(X \times Y)$. Odległość definiujemy wówczas jako 
\begin{align*}
d_2(F, \Phi) = \max_{(x,y) \in X \times Y} \max\left\{\rho(f(x),\phi(x)), \tau(g_x(y), \psi_x(y))\right\} \\ 
= \max\left\{d_1(f,\phi), \max_{x \in X}d_1(g_x, \psi_x)\right\}
\end{align*}
\end{definition}

Uwaga: Rozważmy przestrzeń $(C(X \times Y), d)$ wszystkich ciągłych odwzorowań $X \times Y$ w siebie, z metryką zbieżności jednostajnej: 
$$d(F, \Phi) = \max_{(x,y) \in X \times Y} \max \left\{ \rho\big(f(x,y), \phi(x,y)\big), \tau\big(g(x,y), \psi(x,y)\big) \right\}.$$ 
Wówczas metryka $d_2$ jest równa metryce $d$ obciętej do zbioru $C_\triangle(X \times Y).$

Zbieżność w metrykach $d_1$ i $d_2$ jest tym samym, co zbieżność jednostajna.
Z lematów \ref{przestrzen_ciaglych_jest_zupelna} i \ref{ciagle_trojkatne_tworza_przestrzen_metryczna_zupelna} wynika, że przestrzenie metryczne $(C(X), d_1)$ oraz $(C_\triangle(X \times Y), d_2)$ są zupełne.


\begin{lemma}[Przestrzeń $(C(X), d_1)$ jest zupełna]\label{przestrzen_ciaglych_jest_zupelna}
Niech $(X, \rho)$ będzie zwartą przestrzenią metryczną. Przestrzeń metryczna $(C(X), d_1)$ jest zupełna.
\end{lemma}


\begin{proof}
Musimy pokazać, że każdy ciąg Cauchy'ego w przestrzeni $(C(X), d_1)$ jest zbieżny do elementu tej przestrzeni.
Niech $(f_n)_{n=0}^{\infty} \subset C(X),$ będzie ciągiem Cauchy'ego, czyli 
\begin{equation}
\label{warunek_cauchyego_na_przestrzeni_funkcji_ciaglych}
\forall_{\epsilon > 0} \, \exists_{N \in \mathbb{N}} \, \forall_{n,m > N} \, d_1(f_n, f_m) = \max_{x \in X} \rho\big(f_n(x), f_m(x)\big) < \epsilon.
\end{equation}

Weźmy dowolny $x_0 \in X$, oczywiście 
\begin{equation}
\label{metryka_w_punkcie_mniej_rowna_niz_metryka_zbieznosci_jednostajnej}
\forall_{n,m \in \mathbb{N}} \, \rho\big(f_n(x_0), f_m(x_0)\big) \leq \max_{x \in X} \rho\big(f_n(x), f_m(x)\big).
\end{equation}

Podstawiając \ref{metryka_w_punkcie_mniej_rowna_niz_metryka_zbieznosci_jednostajnej} do \ref{warunek_cauchyego_na_przestrzeni_funkcji_ciaglych} otrzymujemy, że dla każdego $x \in X$ ciąg $\big(f_n(x)\big)_{n=0}^{\infty}$ jest ciągiem Cauchy'ego w przestrzeni $(X, \rho).$ Ponieważ $(X, \rho)$ jest przestrzenią zwartą a więc zupełną (twierdzenie \ref{zwarta_jest_zupelna}), to dla każdego $x \in X$ ciąg $\big(f_n(x)\big)_{n=0}^{\infty}$ jest zbieżny.

Oznaczmy teraz $f(x) = \lim_{n \rightarrow \infty} f_n(x),$ dla każdego $x \in X.$ 


Pozostało pokazać, że ciąg $(f_n)_{n=0}^{\infty} \subset C(X)$ zbiega do $f$ w przestrzeni $(C(X), d_1).$ 
Ustalmy $\epsilon > 0.$ Oczywiście:
$$\exists_{N \in \mathbb{N}} \, \forall_{n,m > N} \, d_1(f_n, f_m) = \max_{x \in X} \rho\big(f_n(x), f_m(x)\big) < \frac{\epsilon}{2}.$$
Z definicji $f$ zachodzi: 
$$\forall_{x \in X} \, \exists_{n_x > N} \, \forall_{n \geq n_x} \, \rho\big(f_{n}(x), f(x)\big) < \frac{\epsilon}{2}.$$
Powyższe nierówności prowadzą do:
$$\forall_{x \in X} \, \forall_{n > N} \rho\big(f_n(x), f(x)\big) \leq \rho\big(f_n(x), f_{n_x}(x)\big) + \rho\big(f_{n_x}(x), f(x)\big) < \frac{\epsilon}{2} + \frac{\epsilon}{2} = \epsilon.$$
Z dowolności $\epsilon$ otrzymujemy:
$$\forall_{\epsilon > 0} \, \exists_{N \in \mathbb{N}} \, \forall_{n > N} \, \forall_{x \in X} \, \rho\big(f_n(x), f(x)\big).$$ 
co jest równoważne następującemu:
$$\forall_{\epsilon > 0} \, \exists_{N \in \mathbb{N}} \, \forall_{n > N} \, \max_{x \in X} \rho\big(f_n(x), f(x)\big) = d_1(f_n, f),$$ 
czyli zbieżności ciągu $(f_n)_{n=0}^{\infty}$ do $f$ w metryce $d_1.$


Zauważmy, że $d_1$ jest metryką zbieżności jednostajnej. Odwzorowanie $f$ jest granicą ciągu odwzorowań ciągłych w metryce zbieżności jednostajnej, a więc $f \in C(X).$
\end{proof}



\begin{lemma}[Ciągłe odwzorowania trójkątne tworzą przestrzeń metryczną zupełną]
\label{ciagle_trojkatne_tworza_przestrzen_metryczna_zupelna}
Niech $(X, \rho)$ i $(Y, \tau)$ będą zwartymi przestrzeniami metrycznymi.
Wówczas przestrzeń metryczna $(C_\triangle(X \times Y), d_2)$ jest zupełna.
\end{lemma}

\begin{proof}
{\color{red} !!!NOWE!!!


Rozważmy przestrzeń $(C(X \times Y), d)$ wszystkich ciągłych odwzorowań postaci: 
$$F: X \times Y \rightarrow X \times Y.$$ 
Weźmy dwa dowolne odwzorowania $F, \Phi \in C(X \times Y)$: 
$$F(x,y) = \big(f(x,y), g(x,y)\big); \, \Phi(x,y) = \big(\phi(x,y), \psi(x,y)\big).$$
Zdefiniujmy metrykę $d$ następująco:
$$d(F, \Phi) = \max_{(x,y) \in X \times Y} \max \left\{ \rho\big(f(x,y), \phi(x,y)\big), \tau\big(g(x,y), \psi(x,y)\big) \right\}$$ 


Pokażemy, że zbiór $(C_\triangle(X \times Y), d_2)$ jest domkniętym podzbiorem $(C(X \times Y), d).$ Weźmy dowolny zbieżny ciąg odwzorowań: $(F_n)_{n=1}^{\infty} \subset C_\triangle(X \times Y).$ Oznaczmy $F = \lim_{n \rightarrow \infty} F_n,$ $\big(F=(f,g), \, F_n=(f_n, g_n)\big).$
Mamy 
$$\forall_{\epsilon > 0} \, \exists_{n_0} \, \forall_{n > n_0} d(F_n, F) = \max_{(x,y) \in X \times Y} \max \left\{ \rho\big(f_n(x,y), f(x,y)\big), \tau\big(g_n(x,y), g(x,y)\big) \right\} < \epsilon,$$
oznacza to, że 
$$\forall_{(x,y) \in X \times Y} \, \lim_{n \rightarrow \infty} \rho\big(f_n(x,y), f(x,y)\big) = \lim_{n \rightarrow \infty} \tau\big(g_n(x,y), g(x,y)\big) = 0.$$
Ponieważ $\forall_{n \in \mathbb{N}} \, F_n \in C_\triangle(X \times Y),$ to $f_n(x,y) = f_n(x).$ 
Mamy więc 
$$\forall_{(x,y) \in X \times Y} \, \lim_{n \rightarrow \infty} \rho\big(f_n(x), f(x,y)\big) = \lim_{n \rightarrow \infty} \rho\big(f_n(x), f_x(y)\big) = 0.$$
{\color{blue}
Ponadto dla każdych $n \in \mathbb{N}, x \in X, y_1, y_2 \in Y$ mamy $f_n(x, y_1) = f_n(x, y_2),$ wobec czego tę samą własność ma odwzorowanie $f.$ Ponieważ $f$ nie zależy od współrzędnej $y$ to odwzorowanie $F$ jest odwzorowaniem trójkątnym, czyli zbiór odwzorowań trójkątnych jest domkniętym podzbiorem przestrzeni $C(X \times Y).$}

Przestrzeń $(C(X \times Y), d)$ jest zupełna na mocy lematu \ref{przestrzen_ciaglych_jest_zupelna}. Ostatecznie $(C_\triangle(X \times Y), d_2)$ jako domknięta podprzestrzeń zupełnej przestrzeni $(C(X \times Y), d)$ również jest przestrzenią zupełną.

}
\end{proof}


\begin{definition}[$\textrm{pr}_1(x, y)$, $\textrm{pr}_2(x, y)$]
Dla $(x, y) \in X \times Y$ niech $\textrm{pr}_1(x, y) = x$ i $\textrm{pr}_2(x, y) = y$. Odwzorowanie identycznościowe na $Y$ będziemy oznaczać przez $\textrm{Id}_Y$ lub krótko $\textrm{Id}$. W dalszej części pracy przestrzeń $Y$ będzie odcinkiem rzeczywistym $I = [0, 1]$.
\end{definition}





\section{Lematy}

Na potrzeby kolejnych lematów wprowadźmy następujące definicje i oznaczenia.
Niech $(X, \rho)$ będzie zwartą przestrzenią metryczną bez punktów izolowanych oraz niech $f \in C(X)$ będzie odwzorowaniem chaotycznym w sensie Devaneya.
$I = [0, 1].$
$P_0$ to dowolny, skończony zbiór zawarty w $X.$ Przez $\mathcal{F}$ oznaczmy zbiór wszystkich odwzorowań $F = (f, g_x)$ ze zbioru $C_\triangle(X \times I)$ spełniających następujące warunki:
\begin{enumerate}
\item $\forall_{x \in X}$ odwzorowanie $g_x$ jest niemalejące i krańce przedziału I pozostawia niezmienione.
\item $\forall_{x \in P_0}$ $g_x$ jest identycznością 
\end{enumerate}


\begin{lemma} \label{F_jest_niepustym_domknietym_podzbiorem_trojkatnych}
Zbiór $\mathcal{F}$ jest niepustym, domkniętym podzbiorem przestrzeni $C_{\triangle}(X \times I)$, 
\end{lemma}

\begin{proof}
Niech $f \in C(X)$ będzie odwzorowaniem chaotycznym w sensie Devaneya. Zdefiniujmy $F_0(x,y) = (f(x), y).$ Dla każdego $x \in X$ odwzorowanie $g_x$ jest identycznością, więc w szczególności jest niemalejące i pozostawia krańce przedziału $I$ niezmienione. Zatem $F_0 \in \mathcal{F},$ czyli $\mathcal{F}$ jest zbiorem niepustym.

Pokazujemy domkniętość. Weźmy dowolny ciąg zbieżny $\big(F_n=(f_n, g_{x,n})\big)_{n=1}^{\infty} \subset \mathcal{F}.$ Oznaczmy jego granicę przez $\widetilde{F}=(\widetilde{f}, \widetilde{g_x}).$
Skoro ciąg ten zawiera się w $\mathcal{F}$ to $\forall_{n \in \mathbb{N}} \, f_n = f.$ Wiemy, że zachodzi 
\begin{equation}
\label{rownanie_granica_ciagu_odwzorowan_ze_zbioru_mathcal_F}
\lim_{n \rightarrow \infty} d_2(F_n, \widetilde{F}) = \lim_{n \rightarrow \infty} \max\{d_1(f, \widetilde{f}), \max_{x \in X} d_1(g_{x,n}, \widetilde{g_x})\} = 0
\end{equation}
Zatem $d_1(f, \widetilde{f}) = 0,$ czyli $\widetilde{f} = f.$

Z równości \ref{rownanie_granica_ciagu_odwzorowan_ze_zbioru_mathcal_F} otrzymujemy również 
$$\lim_{n \rightarrow \infty} \max_{x \in X} d_1(g_{x,n}, \widetilde{g_x}) = 0,$$
a z tego mamy 
$$\forall_{x \in X} \, \lim_{n \rightarrow \infty} d_1(g_{x,n}, \widetilde{g_x}) = 0.$$

Metryka $d_1$ na przestrzeni $C(I)$ jest metryką zbieżności jednostajnej, a wszystkie odwzorowania $g_{x,n}$ są oczywiście ciągłe i monotoniczne, więc dla każdego $x \in X$ odwzorowanie $\widetilde{g_x}$ jest ciągłe i monotoniczne jako granica jednostajnie zbieżnego ciągu monotonicznych odwzorowań ciągłych.
Ponadto, skoro prawdziwy jest warunek 
$$\forall_{x \in X} \, \lim_{n \rightarrow \infty} d_1(g_{x,n}, \widetilde{g_x}) = \lim_{n \rightarrow \infty} \max_{a \in I}\left(g_{x,n}(a), \widetilde{g_x}(a)\right) = 0,$$
to w szczególności zachodzi on dla $a \in \{0, 1\},$ więc skoro odwzorowania $g_{x,n}$ pozostawiają krańce przedziału $I$ niezmienione, to jest to prawdą również dla odwzorowań $\widetilde{g_x}.$ Wiemy, że wszystkie odwzorowania $g_{n,x}$ są monotoniczne. Ustalmy $x_0 \in P_0,$ wszystkie odwzorowania $g_{x_0,n}$ są identycznością, wówczas $\widetilde{g_{x_0}} = \textrm{Id}$ jako granica ciągu stałego. Odwzorowania $\widetilde{g_x}$ spełniają więc warunki narzucane przez definicję zbioru $\mathcal{F}.$
Zatem granica $\widetilde{F}$ należy do $\mathcal{F},$ czyli dowolny, ustalony ciąg zbiega do elementu zbioru $\mathcal{F},$ więc zbiór ten jest domknięty.
\end{proof}


\begin{theorem} \label{tranzytywne_rezydualne_w_F}
Zbiór odwzorowań tranzytywnych jest rezydualny w $\mathcal{F}$.
\cite{alseda1999entropy} - dowod twierdzenia 1.5
\end{theorem}

\begin{proof}
\textcolor{red}{TODO: dowód jest w pracy {alseda1999entropy} - dowod twierdzenia 1.5}
\end{proof}



\begin{definition} \label{Fso_sa_otwartymi_podzbiorami_F}
Niech $U$ będzie dowolnym zbiorem otwartym zawartym w $X \times I$. Zdefiniujmy zbiór $\mathcal{F}_{SO}$ (,,S'' - stabilny, ,,O'' - okresowy) następująco. Odwzorowanie $G$ należy do $\mathcal{F}_{SO}$ wtedy i tylko wtedy, gdy należy do $\mathcal{F}$, posiada punkt okresowy w $U$ oraz wszystkie dostatecznie bliskie $G$ odwzorowania z $\mathcal{F}$ również posiadają punkt okresowy w $U$ (być może różne od punktów okresowych odwzorowania $G$).
\end{definition}
Bezpośrednio z tej definicji wynika, że $\mathcal{F}_{SO}$ jest otwartym podzbiorem $\mathcal{F}.$




\begin{lemma}
\label{w_otoczeniu_niezmienniczego_jest_otwarty_zawarty_w_iteracjach}

Niech $(X, \rho)$ będzie zwartą przestrzenią metryczną bez punktów izolowanych, $f \in C(X)$ odwzorowaniem chaotycznym w sensie Devaneya, $P$ skończonym zbiorem niezmienniczym ze względu na $f$, a $V$ otwartym otoczeniem zbioru $P$.
Wówczas dla każdego $N \in \mathbb{N}$ istnieje niepusty zbiór otwarty $W \subseteq V,$ taki że $W \cup f(W) \cup \ldots \cup f^N(W) \subseteq V$.
\end{lemma}

\begin{proof}
$P$ jest zbiorem skończonym, więc możemy wypisać wszystkie jego elementy: 
$$P = \{p_1, p_2, \ldots, p_k\}$$
Ponieważ $V$ jest otwartym otoczeniem zbioru $P$ to istnieje $d>0,$ takie że dla każdego ${0<\epsilon<d}$ zbiór $\bigcup_{i=0}^k K(p_i, \epsilon)$ zawiera się w $V.$ Ustalmy $0<\epsilon<d$ i oznaczmy $U=\bigcup_{i=0}^k K(p_i, \epsilon),$ oczywiście $U \subseteq V.$ Rozważmy dowolny, ustalony punkt $p_i$ ze zbioru $P.$ Wiemy, że $P$ jest niezmienniczy ze względu na odwzorowanie $f$, a więc istnieją punkty $p_{j_1}, p_{j_2}, \ldots, p_{j_N} \in P$ dla których zachodzi $f(p_i) = p_{j_1}, \, f^2(p_i) = p_{j_2}, \, \ldots, \, f^N(p_i) = p_{j_N}.$ Z kolei z ciągłości odwzorowania $f$, również jego iteraty są ciągłe, a z tego wynika, że znajdziemy taką $\delta_i,$ że $\epsilon >\delta_i > 0$ i dla której $f^n(K(p_i, \delta_i)) \subset K(p_{j_n}, \epsilon) \subset U,$ dla każdego $n \in \{1,2,\ldots,N\}.$ Z dowolności wyboru $p_i$ otrzymujemy, że każdy zbiór postaci $f^n(K(p_i, \delta_i)),$ (gdzie $n \in \{1,2,\ldots,N\}$ a $p_i \in P$) zawiera się w $U$. Oznaczmy $W = \bigcup_{i=1}^k K(p_i, \delta_i),$ oczywiście $W$ jest niepustym zbiorem otwartym, który zawiera się w $U$ ponieważ $\delta_i < \epsilon$ dla każdego $i \in \{1,2,\ldots,k\}.$ Ponadto
$$\forall_{n \in \{1,2,\ldots,k\}} \, f^n(W) = f^n\left(\bigcup_{i=1}^k K(p_i, \delta_i)\right) = \bigcup_{i=1}^k f^n\big(K(p_i, \delta_i)\big) \subset U.$$
Ostatecznie otrzymujemy $W \cup f(W) \cup \ldots \cup f^N(W) \subset U \subseteq V.$
\end{proof}



\begin{lemma}[Istnienie orbity okresowej odwiedzającej rodzinę zbiorów otwartych]
\label{lemat_3_glownego_artykulu_istnieje_orbita_okresowa_krojaca_sie_z_rodzina_otwartych}

Na podstawie \cite[s.~231-232 Lemma 3]{balibrea2003topological} i \cite[s.~7]{someAspectsOfTopologicalTransitivity}
Niech $(X, \rho)$ będzie zwartą przestrzenią metryczną.
Niech $f \in C(X)$ będzie topologicznie tranzytywnym odwzorowaniem, którego zbiór punktów okresowych jest gęsty w $X$. 

Wtedy dla każdej rodziny niepustych zbiorów otwartych $U_1, U_2, \ldots, U_n \subset X$ istnieje orbita okresowa $f$, której przekrój z każdym ze zbiorów $U_i, i=1,2,\ldots,n$ jest niepusty
\end{lemma}

\begin{proof}
Ponieważ odwzorowanie $f$ jest tranzytywne a przestrzeń $X$ jest zwarta, to istnieje punkt $x_0$, którego orbita jest gęsta w $X$. Zatem orbita punktu $x_0$ odwiedza każdy ze zbiorów otwartych $U_i, \, i = 1,2,\ldots,n.$ Teraz korzystając z ciągłości odwzorowania $f$ wystarczy wziąć punkt okresowy $f$ dostatecznie bliski $x_0.$
\end{proof}



\begin{lemma}
\label{lemat_4_glownego_artykulu}
\cite{alseda1999entropy}
Niech $(X, \rho)$ będzie zwartą przestrzenią metryczną i niech $F = (f, g_x)$ będzie odwzorowaniem należącym do $C_\triangle(X \times I)$ którego wszystkie odwzorowania włóknowe są niemalejące i pozostawiają krańce $I$ niezmienione. Niech $\{a_1, a_2, \ldots, a_n\}$ będzie podzbiorem $X$ oraz dla $i = 1, 2, \ldots, n$ niech $U_i$ będą parami rozłącznymi zbiorami otwartymi, takimi że $a_i \in U_i$. Załóżmy, że $h_i$ są niemalejącymi odwzorowaniami z $C(I)$ pozostawiającymi krańce $I$ niezmienione i spełniającymi $d_1(h_i, g_{a_i}) < \epsilon$ dla pewnego dodatniego $\epsilon$ i każdego $i = 1, 2, \ldots, n$. Wówczas istnieje odwzorowanie 
$$\widetilde{F} = (f, \widetilde{g}_x) \in C_\triangle(X \times I)$$ spełniające cztery następujące warunki:

\begin{enumerate}
\item  wszystkie odwzorowania włóknowe $\widetilde{F}$ są niemalejące i pozostawiające krańce $I$ niezmienione,
\item $d_2(F, \widetilde{F}) < \epsilon$,
\item $\widetilde{g}_{a_i} = h_i$ dla $i = 1,2,\ldots,n$,
\item $\widetilde{g}_x = g_x$ dla $x \in X \setminus \bigcup_{i=1}^n U_i$.
\end{enumerate}

\begin{proof}
\cite{alseda1999entropy}
Dla każdego $i = 1, 2, \ldots, n$ niech $V_i \subset U_i$ będzie otwartym otoczeniem $a_i,$ takim że dla pewnego dodatniego $\widetilde{\epsilon} < \epsilon$, $d_1(h_i, g_x) < \widetilde{\epsilon}$ zawsze wtedy gdy $x \in V_i$.

Oznaczmy $U = \bigcup_{i=1}^n U_i$, $V = \bigcup_{i=1}^n V_i$. Niech $u : X \longrightarrow [0,1]$ \textcolor{red}{TODO: cwiczenie - dlaczego funkcja u istnieje?} będzie ciągłą funkcją, przyjmującą wartość $1$ na zbiorze $\{a_1, a_2, \ldots, a_n\}$, natomiast $0$ poza zbiorem $V$. Zastąpmy każde odwzorowanie włóknowe $g_x$ przez $\widetilde{g}_x$, gdzie

\begin{equation} \label{lemat4_rownanie_1}
    \widetilde{g}_x(y) =
    \begin{cases}
        g_x(y) & \text{dla $x \in X \setminus V$,}\\
        g_x(y)(1-u(x))+h_i(y)u(x) & \text{dla $x \in V_i : i \in \{1,2,\ldots,n\}$.}
    \end{cases}
\end{equation}

\textcolor{red}{TODO: Przemyslec, w kontekscie powyzszego zdanie: kombinacja wypukla odwzorowan niemalejacych jest niemalejaca - do czego to tutaj jest porzebne}

dla każdego $y \in I$. Zauważmy ponadto, że dla $x \in V_i$ oraz $i \in \{1,2,\ldots,n\}$ możemy równoważnie napisać

\begin{equation} \label{lemat4_rownanie_2}
\widetilde{g}_x(y) = u(x)(h_i(y) - g_x(y)) + g_x(y).
\end{equation}

Rozważmy odwzorowanie $\widetilde{F} = (f, \widetilde{g}_x)$. Należy ono do $C_\triangle(X \times I)$. Z równości \ref{lemat4_rownanie_1} widzimy, że wszystkie odwzorowania włóknowe $\widetilde{g}_x$ są niemalejące i krańce przedziału $I$ są ich punktami stałymi. Ponadto $\widetilde{g}_{a_i} = h_i$ dla każdego $i$ oraz $\widetilde{g}_x = g_x$ dla $x \in X \setminus V \supset X \setminus U$. Ponieważ dla $x \in V_i$, gdzie $i \in \{1,2,\ldots,n\}$ mamy $d_1(h_i, g_x) < \widetilde{\epsilon}$ oraz $u(x) \in [0,1]$, zatem z równości \ref{lemat4_rownanie_2} otrzymujemy $d_1(g_x, \widetilde{g}_x) < \widetilde{\epsilon}$ dla każdego $x \in V$. {\color{red} !!!NOWE!!!
Wynika z tego, że 
\begin{align*}
d_2(F, \widetilde{F}) & = d_2\big((f, g_x), (f, \widetilde{g_x})\big) \\
& = \max\left\{d_1(f, f), \max_{x \in X}d_1(g_x, \widetilde{g_x})\right\} \\
& = \max\left\{0, \max_{x \in X}d_1(g_x, \widetilde{g_x})\right\} \\
& = \max_{x \in X}d_1(g_x, \widetilde{g_x}) \\
& \leq \widetilde{\epsilon} < \epsilon,
\end{align*}
co kończy dowód.
}
\end{proof}

\end{lemma}



\chapter{Twierdzenie o rozszerzaniu odwzorowań chaotycznych w sensie Devaneya}
\begin{theorem}[O rozszerzaniu]\label{twierdzenie_glowne}
Twierdzenie wraz z dowodem przytaczamy za pracą \cite{balibrea2003topological}
Niech $(X, \rho)$ będzie zwartą przestrzenią metryczną bez punktów izolowanych oraz niech $f \in C(X)$ będzie odwzorowaniem chaotycznym w sensie Devaneya. Wówczas odwzorowanie $f$ można rozszerzyć do odwzorowania $F \in C_{\triangle}(X \times I)$ (to znaczy tak, że $f$ jest odwzorowaniem bazowym dla $F$) w taki sposób, że:
\begin{enumerate}[label=(\roman*)]
\item\label{twierdzenie_glowne_a} $F$ jest również chaotyczne w sensie Devaneya, 
\item\label{twierdzenie_glowne_b} $F$ ma taką samą entropię topologiczną jak $f$, 
\item\label{twierdzenie_glowne_c} zbiory $X \times \{0\}$ i $X \times \{1\}$ są niezmiennicze ze względu na $F$.
\end{enumerate}
\end{theorem}


\begin{proof}[Dowód twierdzenia o rozszerzaniu]
Odwzorowanie $f$ jest chaotyczne w sensie Devaneya, zatem ma gęsty zbiór punktów okresowych, w szczególności istnieje orbita okresowa.
Możemy zatem ustalić okresową orbitę $P_0$ odwzorowania $f$. Ponieważ $P_0$ jest zbiorem skończonym a $X$ nie ma punktów izolowanych, to $P_0$ jest nigdziegęstym domkniętym podzbiorem $X$.

Rozważmy zbiór $\mathcal{F}$ wszystkich odwzorowań $F = (f, g_x)$ ze zbioru $C_\triangle(X \times I)$ spełniających następujące warunki:
\begin{enumerate}
\item\label{dowod_glowny_a} Odwzorowanie bazowe $f$ jest odwzorowaniem z założenia twierdzenia \ref{twierdzenie_glowne}.
\item\label{dowod_glowny_b} $\forall_{x \in X}$ odwzorowanie $g_x$ jest niemalejące i krańce przedziału I pozostawia niezmienione.
\item\label{dowod_glowny_c} $\forall_{x \in P_0}$ $g_x$ jest identycznością 
\end{enumerate}

Warunek \ref{dowod_glowny_b} implikuje, że dla każdego odwzorowania z $\mathcal{F}$ zbiory $X \times \{0\}$ i $X \times \{1\}$ są niezmiennicze, czyli $\forall_{F \in \mathcal{F}}$ zachodzi warunek \ref{twierdzenie_glowne_c}  twierdzenia \ref{twierdzenie_glowne}. Zachodzenie warunku \ref{twierdzenie_glowne_b} twierdzenia \ref{twierdzenie_glowne} dla każdego odwzorowania $F \in \mathcal{F}$ wynika z faktu, że każde odwzorowanie z $C_\triangle(X \times I)$, gdzie $X$ jest zwartą przestrzenią metryczną (a więc w szczególności każde odwzorowanie z $\mathcal{F}$), którego wszystkie odwzorowania włóknowe są niemalejące, ma taką samą entropię topologiczną jak jego odwzorowanie bazowe (twierdzenie \ref{rownosc_entropii_gdy_wlokna_monotoniczne}).
Pozostaje zatem wykazać prawdziwość warunku \ref{twierdzenie_glowne_a}, czyli chaotyczność w sensie Devaneya jakiegoś odwzorowania $F \in \mathcal{F}$. Takie odwzorowanie będzie bowiem łącznie spełniało wszystkie 3 warunki, czyli tezę twierdzenia.

Z lematu \ref{warunki_dostateczne_chaotycznosci_devaneya} wynika, że aby odwzorowanie $F$ było chaotyczne w sense Devaneya potrzeba i wystarcza, żeby spełniało dwa poniższe warunki:
\begin{enumerate}
\item \label{devaney_pierwsza_wlasnosc} $F$ jest topologicznie tranzytywne
\item \label{devaney_druga_wlasnosc} zbiór punktów okresowych odwzorowania $F$ jest gęsty w $(X \times I)$
\end{enumerate}
Jest tak gdyż przestrzeń $(X \times I)$ spełnia założenia lematu, tj. jest przestrzenią nieskończoną i zwartą.

Chcemy wykazać, że istnieje $F \in \mathcal{F}$ będące jednocześnie topologicznie tranzytywne i posiadające gęsty w $(X \times I)$ zbiór punktów okresowych. Z lematu \ref{F_jest_niepustym_domknietym_podzbiorem_trojkatnych} wiemy, że $\mathcal{F}$ jest niepustym, domkniętym podzbiorem przestrzeni $C_{\triangle}(X \times I)$, która jak wynika z lematu \ref{ciagle_trojkatne_tworza_przestrzen_metryczna_zupelna} jest przestrzenią metryczną zupełną. 

Przekrój dwóch zbiorów rezydualnych jest rezydualny (patrz lemat \ref{przekroj_rezydualnych_jest_rezydualny}) a więc niepusty. Wystarczy zatem pokazać, że oba zbiory:
\begin{itemize}
\item zbiór odwzorowań topologicznie tranzytywnych
\item zbiór odwzorowań, których zbiór punktów okresowych jest gęsty w $(X \times I)$
\end{itemize}
są rezydualne w $\mathcal{F}$. Wówczas każde odwzorowanie należące do ich przekroju będzie spełniało wszystkie trzy warunki tezy twierdzenia \ref{twierdzenie_glowne} 

Zbiór odwzorowań tranzytywnych jest rezydualny w $\mathcal{F}$, zostało to udowodnione w twierdzeniu \ref{tranzytywne_rezydualne_w_F}

Pozostało wykazać, że również zbiór odwzorowań posiadających gęsty zbiór punktów okresowych jest rezydualny w $\mathcal{F}$.
Oznaczmy zbiór takich odwzorowań (jednocześnie należących do $\mathcal{F}$) przez $\mathcal{F}_{DP}$.

Niech $\{U_i^X\}_{i=1}^{\infty}$ będzie bazą topologii $X$ i niech $\{U_i^I\}_{i=1}^{\infty}$ będzie zbiorem wszystkich odcinków otwartych o końcach wymiernych, należących do odcinka otwartego $(0, 1)$. Niech $\{U_i\}_{i=1}^{\infty}$ będzie ponumerowaniem zbioru $\{U_i^X \times U_j^I : i,j \in \mathbb{N}\}$. Wtedy każda kula otwarta w $X \times I$ zawiera jakiś spośród otwartych zbiorów $U_i$.

Dla każdego $i=1,2,...$ niech zbiór $\mathcal{F}_{SO}^i$ (,,S'' - stabilny, ,,O'' - okresowy) będzie zdefiniowany następująco. Odwzorowanie G należy do $\mathcal{F}_{SO}^i$ wtedy i tylko wtedy, gdy należy do $\mathcal{F}$, posiada punkt okresowy w $U_i$ oraz wszystkie dostatecznie bliskie $G$ odwzorowania z $\mathcal{F}$ również posiadają punkt okresowy w $U_i$ (być może różne od punktów okresowych odwzorowania $G$). Zbiory $\mathcal{F}_{SO}^i$ są otwartymi podzbiorami $\mathcal{F}$ (patrz lemat \ref{Fso_sa_otwartymi_podzbiorami_F}). Ponieważ $\mathcal{F}_{DP} \supseteq \bigcap_{i=1}^{\infty} \mathcal{F}_{SO}^i$ aby pokazać, że $\mathcal{F}_{DP}$ jest rezydualny w $\mathcal{F}$ wystarczy pokazać, że $\forall_{i \in \mathbb{N}} \, \mathcal{F}_{SO}^i$ jest gęsty w $\mathcal{F}$. (Wynika to z twierdzenia Baire'a \ref{twierdzenie_bairea} oraz definicji \ref{definicja_zbioru_rezydualnego}).

Aby wykazać że każdy zbiór $\mathcal{F}_{SO}^i$ jest gęsty w $\mathcal{F}$ ustalmy dowolne: $i \in \mathbb{N}$, $F=(f,g_x) \in \mathcal{F}$ i $\epsilon > 0$. Pokażemy, że istnieje odwzorowanie $G \in \mathcal{F}_{SO}^i$, którego odległość od $F$ nie przekracza $\epsilon$. Dla uproszczenia sytuacji załóżmy, że $\rho(\textrm{pr}_1(U_i), P_0) > 0$ (Jeżeli tak nie jest, zawsze możemy wziąć zamiast $U_i$ mniejszy prostokąt $U_i^* \subset U_i$).

Weźmy dodatnią liczbę naturalną $N \geq \frac{4}{\epsilon}$. Następnie rozważmy otwarte otoczenie $V$ orbity $P_0$ w przestrzeni $(X, \rho),$ takie że $\rho(\textrm{pr}_1(U_i), V) \geq 0$ oraz $d_1(g_x, \textrm{Id}) < \frac{\epsilon}{4}$ dla każdego $x \in V$ (pamiętamy, że $g_x = \textrm{Id}$ dla $x \in P_0$).

$P_0$ jest skończonym zbiorem niezmienniczym ze względu na $f$, więc na mocy lematu \ref{w_otoczeniu_niezmienniczego_jest_otwarty_zawarty_w_iteracjach} istnieje niepusty zbiór otwarty $W \subseteq V,$ taki że $W \cup f(W) \cup \ldots \cup f^N(W) \subseteq V$. Na mocy lematu \ref{lemat_3_glownego_artykulu_istnieje_orbita_okresowa_krojaca_sie_z_rodzina_otwartych} istnieje punkt okresowy $x_0$ odwzorowania $f,$ taki że $x_0 \in \textrm{pr}_1(U_i)$ oraz orbita $x_0$ ma niepusty przekrój ze zbiorem $W$. Niech $r > 0$ będzie pierwszą dodatnią liczbą całkowitą dla której $f^r(x_0) \in W$. Wtedy $f^r(x_0), f^{r+1}(x_0), \ldots, f^{r+N-1}(x_0) \in V$. Niech $s \geq 0$ będzie pierwszą nieujemną liczbą całkowitą dla której $f^{r+N+s}(x_0) = x_0$, tj. $r+N+s$ jest okresem punktu $x_0$. Weźmy $y_0,$ takie że $(x_0, y_0) \in U_i$. Ponieważ wszystkie odwzorowania włóknowe $(g_x)$ odwzorowania $F$ są ,,na'', to istnieje punkt $y^* \in (0,1),$ taki że $F^s(f^{r+N}(x_0), y^*) = (x_0, y_0)$. Uzasadnienie:


\begin{equation}
\begin{aligned}
F^s\left(f^{r+N}(x_0), y^*\right) & = \Big( f^{r+N+s}(x_0), \, g_{x_s}\big(g_{x_{s-1}}\big( \ldots \big(g_{x_1}(y^*)\big) \ldots \big)\big) \Big) \\
& = \Big( x_0, \, \big(g_{x_s} \circ g_{x_{s-1}} \circ \ldots \circ g_{x_1} \big)(y^*) \Big),
\end{aligned}
\end{equation}
gdzie $x_1 = f^{r+N}(x_0), \, x_{k+1} = f(x_k)$ dla $k=1,2,\ldots,s-1.$
Oznaczmy  $\big(g_{x_s} \circ g_{x_{s-1}} \circ \ldots \circ g_{x_1} \big)$ przez $g_s.$ Funkcja $g_s$ jest ,,na'' jako złożenie funkcji ,,na''. W związku z tym, dla każdego $z \in [0,1]$ istnieje $z^* \in [0,1],$ takie że $g_s(z^*) = z.$ Ponadto jeżeli $z \in (0,1)$, to $z^* \in (0,1)$ bo $g_s$ pozostawia krańce przedziału $[0,1]$ niezmienione.



Przypadek 1. $z = \textrm{pr}_2(F^r(x_0, y_0))$ jest różne od 0 i 1.
Oznaczmy przez $g$ odwzorowanie z $C(I)$ posiadające następujące trzy własności:
\begin{enumerate}[label=(g\arabic*)]
\item \label{g_jeden} $d_1(g, \mathrm{Id}) < \frac{\epsilon}{4}$,
\item \label{g_dwa} $g$ jest odwzorowaniem niemalejącym, pozostawiającym końce przedziału $I$ niezmienione,
\item \label{g_trzy} $g^N(z) = y^*$.
\end{enumerate}

Następnie, rozważmy odwzorowanie $h \in C(I)$ posiadające trzy następujące własności:
\begin{enumerate}[label=(h\arabic*)]
\item \label{h_jeden} $d_1(h, g_{x_0}) < \frac{\epsilon}{4}$,
\item \label{h_dwa} $h(y_0) = g_{x_0}(y_0)$,
\item \label{h_trzy} $h$ jest stałe na zwartym odcinku $[a,b] \subseteq \textrm{pr}_2(U_i)$ zawierającym punkt $y_0$ w swoim wnętrzu.
\end{enumerate}

Weźmy teraz odwzorowanie $G = (f, \widetilde{g}_x) \in \mathcal{F},$ takie że $d_2(G, F) < \frac{\epsilon}{2}$ oraz

\begin{equation}
\label{rownanie_g_z_falka}
    \widetilde{g}_x =
    \begin{cases}
        h & \text{if $x=x_0$,}\\
        g & \text{if $x \in \left\{f^k(x_0) : r \leq k \leq r+N-1\right\}$,}\\
        g_x & \text{if $x \in \left\{f^k(x_0) : 1 \leq k \leq r-1 \lor r+N \leq k \leq r+N+s-1\right\}$.}
    \end{cases}
\end{equation}

Odwzorowanie takie istnieje na mocy lematu \ref{lemat_4_glownego_artykulu}, ponieważ
$$d_1(\widetilde{g}_{x_0}, g_{x_0}) = d_1(h, g_{x_0}) < \frac{\epsilon}{4}$$
oraz dla $x \in \left\{f^r(x_0), f^{r+1}(x_0), \ldots, f^{r+N-1}(x_0)\right\}$,
$$d_1(\widetilde{g}_x, g_x) = d_1(g, g_x) \leq d_1(g, \mathrm{Id}) + d_1(\mathrm{Id}, g_x) < \frac{\epsilon}{4} + \frac{\epsilon}{4} = \frac{\epsilon}{2}$$.

Punkt $(x_0, y_0) \in U_i$ jest punktem okresowym odwzorowania $G$, ponieważ

\begin{equation}
\begin{split}
G^{r+N+s}\left(x_0, y_0\right) & \stackrel{(1)}{=} G^{r+N+s-1}\big(G(x_0, y_0)\big) \\
& \stackrel{(2)}{=} G^{r+N+s-1}\big(F(x_0, y_0)\big) \stackrel{(3)}{=} G^{N+s}\big(F^r(x_0, y_0)\big) \\
& \stackrel{(4)}{=} G^{N+s}\big(f^r(x_0), z\big) \stackrel{(5)}{=} G^s\big(f^{r+N}(x_0), y^*\big) \\
& \stackrel{(6)}{=} F^s\big(f^{r+N}(x_0), y^*\big) \stackrel{(7)}{=} (x_0, y_0).
\end{split}
\end{equation}

Szczegółowe uzasadnienie powyższych równości:
\begin{enumerate}
\item Nie musimy rozważać odwracalności $G$ gdyż $r > 0, \, N > 0, \, s \geq 0$, a więc $r+N+s \geq 2.$
\item $G = (f, \widetilde{g}_x), \, \widetilde{g}_{x_0} = h$ a $h(y_0) = g_{x_0}(y_0).$
\item Z równania \ref{rownanie_g_z_falka} mamy $\widetilde{g}_x = g_x,$ dla $x \in \{f^k(x_0) : 1 \leq k \leq r-1\}.$
\item Wynika z definicji $z.$
\item Z równania \ref{rownanie_g_z_falka} dla $x \in \{f^k(x_0) : r \leq k \leq r+N-1\},$ oraz \ref{g_trzy}.
\item Z równania \ref{rownanie_g_z_falka} mamy $\widetilde{g}_x = g_x,$ dla $x \in \{f^k(x_0) : r+N \leq k \leq r+N+s-1\}.$
\item Wprost z definicji $y^*.$
\end{enumerate}

Ponadto, ze względu na \ref{h_trzy} zachodzi

\begin{equation}
\begin{aligned}
G^{r+N+s}(\{x_0\} \times [a, b]) & = G^{r+N+s-1}\big(G(\{x_0\} \times [a, b])\big) \\
& = G^{r+N+s-1}\big(\{f(x_0)\} \times \widetilde{g}_x([a,b])\big) \\
& \stackrel{\ref{h_trzy}}{=} G^{r+N+s-1}\big(\{f(x_0)\} \times \widetilde{g}_{x_0}([a,b])\big) \\
& \stackrel{\ref{h_dwa}}{=} G^{r+N+s-1}\big(\{f(x_0)\} \times \{g_{x_0}(y_0)\}\big) \\
& = G^{r+N+s-1}\big(\{(f(x_0), g_{x_0}(y_0))\}\big) \\
& = G^{r+N+s-1}\big(\{F(x_0, y_0)\}\big) \\
& = \{(x_0, y_0)\}.
\end{aligned}
\end{equation}


Z faktu $y_0 \in (a, b)$ wynika, że każde odwzorowanie $\widetilde{G} \in \mathcal{F}$ dostatecznie bliskie $G$ posiada własność
$$\widetilde{G}^{r+N+s}(\{x_0\} \times [a, b]) \subseteq \{x_0\} \times [a, b].$$

$\widetilde G^{r+N+s}$ na pierwszej współrzędnej przeprowadza $x_0$ na $x_0$, zaś na drugiej przeprowadza odcinek $[a,b]$ w $[a,b]$. Każde ciągłe odwzorowanie odcinka w siebie ma punkt stały, a punkt stały iteraty jest punktem okresowym oryginalnego odwzorowania.
Zatem $\widetilde{G}$ posiada punkt okresowy w $\{x_0\} \times [a, b] \subseteq U_i$.

Zatem $G \in \mathcal{F}_{SO}^i$, co kończy dowód dla przypadku 1.

Przypadek 2. $\textrm{pr}_2(F^r(x_0, y_0))$ jest równy $0$ lub $1$.

W takim przypadku użyjemy lematu \ref{lemat_4_glownego_artykulu} aby dostać odwzorowanie $H = (f, h_x) \in \mathcal{F},$ takie że $d_2(H, F) < \frac{\epsilon}{2}$ i dla $x \in \{x_0, f(x_0), \ldots, f^{r-1}(x_0)\}$ odwzorowania włóknowe $h_x$ są ściśle rosnące. Takie odwzorowanie istnieje, ponieważ każdą funkcję niemalejącą możemy jednostajnie przybliżać funkcjami ściśle rosnącymi.

 
Ponieważ $y_0$ jest różnie od $0$ i $1$ dostajemy, że $\textrm{pr}_2(H^r(x_0, y_0))$ również jest różne od $0$~i~$1$. (Ponieważ $H = (f, h_x) \in \mathcal{F}$, to każde z odwzorowań $h_x$ pozostawia niezmienione punkty $0$ i $1$. Wprowadźmy oznaczenie $\textrm{pr}_2(H^r(x_0, y_0)) = (h_{f^{r-1}(x_0)} \circ \ldots \circ h_{f(x_0)} \circ h_{x_0})(y_0) = h_0(y_0)$. Odwzorowanie $h_0$ jako złożenie odwzorowań ściśle rosnących i pozostawiających punkty $0$ i $1$ niezmienione jest ściśle rosnące i pozostawia niezmienione punkty $0$ i $1$.  Ponieważ $y_0 > 0$ to oczywiście $h_0(y_0) > h_0(0) = 0$. Natomiast ze względu na fakt, że $y_0 < 1$, to zachodzi $h_0(y_0) < h_0(1) = 1$.)
 
Następnie korzystając z przypadku 1 dostajemy odwzorowanie $G \in \mathcal{F}_{SO}^i$, dla którego zachodzi nierówność $d_2(G, H) < \frac{\epsilon}{2}$. Wówczas, korzystając z nierówności trójkąta otrzymujemy $d_2(G, F) \leq d_2(G, H) + d_2(H, F) < \frac{\epsilon}{2} + \frac{\epsilon}{2} = \epsilon$.


\end{proof}




% END MOJE ====================================================



%%%%%%%%%%%%%%%%%%%%%%%%%%%%%%%%%%%%%%%%%%%%%%%%%%%%%%%%%
% BIBLIOGRAFIA
% W tworzeniu bibliografii najlepiej korzystać z BibTex'a, 
% który jest częścią systemu Tex. W naszym przypadku funkcję 
% przechowalni literatury, do której się odwołujemy, pełni 
% plik bibliografia.bib. Nie musimy ręcznie dodawać nowych 
% pozycji do bibliografii. Możemy wejść np. na stronę 
% https://mathscinet.ams.org/mathscinet/index.html, 
% znaleźć odpowiednią pozycję, wybrać ją, a następnie zmienić 
% 'Select alternative format' na BibTeX, skopiować uzyskany 
% tekst, wkleić do pliku bibliografia.bib i skompilować. 
% Gotowe informacje do pliku bibliografia.bib można znaleźć 
% także na https://arxiv.org - gdy znajdziemy interesującą nas 
% pracę, szukamy 'References & Citations' i klikamy 'NASA ADS', 
% a potem 'Bibtex entry for this abstract' 
% i postępujemy tak jak wcześniej.
%%%%%%%%%%%%%%%%%%%%%%%%%%%%%%%%%%%%%%%%%%%%%%%%%%%%%%%%%
\newpage
% w nawiasie klamrowym wpisujemy nazwę pliku z bibliografią w formacie .bib
\bibliografia{bibliografia.bib} 
\end{document}