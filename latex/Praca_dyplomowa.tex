%%%%%%%%%%%%%%%%%%%%%%%%%%%%%%%%%%%%%%%%%%%%%%%%%%%%%%%%%
% Niniejszy plik przedstawia przykładowy skład 
% pracy dyplomowej na Wydziale Matematyki PWr. 
% 
% Autorzy: 
% Damian Fafuła
% Michał Kijaczko
% Jakub Michalczak
% Maciej Miśta
% Dagmara Nowak
% Tomasz Skalski
% Wojciech Słomian
%
%% Data utworzenia: 8.05.2018
% Numer wersji: 1
%
% Poniższą formatkę można rozpowszechniać i edytować 
% pod warunkiem zachowania numeru wersji, 
% informacji o autorach i dodaniu informacji 
% o wprowadzonych zmianach.
%
%%%%%%%%%%%%%%%%%%%%%%%%%%%%%%%%%%%%%%%%%%%%%%%%%%%%%%%%%
% Domyślną opcją jest: praca magisterska, język polski.
% W przypadku pracy pisanej w języku angielskim dodajemy 
% opcję [english].
% Dla pracy licencjackiej dodajemy opcję [licencjacka].
% Dla pracy inżynierskiej dodajemy opcję [inzynierska].
% Dopuszczalne są podwójne opcje, np. [licencjacka, english].
% Opcje dodajemy w kwadratowym nawiasie przy \documentclass.
%
%
%%%%%%%%%%%%%%%%%%%%%%%%%%%%%%%%%%%%%%%%%%%%%%%%%%%%%%%%%
\documentclass[licencjacka]{pwr_wmat_praca_dyplomowa}

\usepackage{enumitem}
\usepackage{xcolor}
%%%%%%%%%%%%%%%%%%%%%%%%%%%%%%%%%%%%%%%%%%%%%%%%%%%%%%%%%
%              DANE DO PRACY
%
% W przypadku pracy dyplomowej w języku angielskim nie jest konieczne 
% wypełnianie pól: \tytul{}, \kierunek{}, \specjalnosc{}, 
%                  \streszczenie{}, \slowakluczowe{}.
%%%%%%%%%%%%%%%%%%%%%%%%%%%%%%%%%%%%%%%%%%%%%%%%%%%%%%%%%
%
% Imię i nazwisko autora
\autor{Imię i nazwisko dyplomanta}
%
% Tytuł pracy dyplomowej 
\tytul{Tytuł pracy dyplomowej} 
\tytulang{Tytuł pracy dyplomowej w języku angielskim}
%
% Tytuł / stopień / imię i nazwisko opiekuna
\opiekun{dr inż. Dawid Huczek}
%
% Kierunek studiów wybieramy spośród następujących:
% 1) Matematyka
% 2) Matematyka i Statystyka
% 3) Matematyka stosowana
\kierunekstudiow{Matematyka}
%
% Kierunek studiów po angielsku wybieramy spośród następujących:
% 1) Mathematics
% 2) Mathematics and Statistics
% 3) Applied Mathematics
\kierunekstudiowang{Mathematics}
%
% Specjalność wybieramy spośród następujących: 
% KIERUNEK: Matematyka
% 1) Matematyka teoretyczna,
% 2) Statystyka matematyczna,
% 3) Matematyka finansowa i ubezpieczeniowa,
%
% KIERUNEK: Matematyka i Statystyka
% 4) Matematyka,
% 5) Statystyka i analiza danych, 
%
% 6) -- (w przypadku braku specjalizacji).
\specjalnosc{Matematyka teoretyczna} 
%
% Specjalność w języku angielskim wybieramy spośród następujących:
% KIERUNEK: Matematyka
% 1) Theoretical Mathematics,
% 2) Mathematical Statistics,
% 3) Financial and Actuarial Mathematics,
%
% KIERUNEK: Matematyka i Statystyka
% 4) Mathematics,
% 5) Statistics and Data Analysis,
%
% KIERUNEK: Applied Mathematics
% 6) Financial and Actuarial Mathematics, 
% 7) Mathematics for Industry and Commerce,
% 8) Computational Mathematics,
% 9) Modelling, Simulation and Optimization.
%
% 10) -- (w przypadku braku specjalizacji).
\specjalnoscang{Theoretical Mathematics} 
%
% Krótkie streszczenia po polsku i angielsku
% - nie dłuższe niż 530 znaków.
\streszczenie{Tutaj piszemy krótkie streszczenie pracy (nie powinno być dłuższe niż 530 znaków).}
\streszczenieang{Tutaj piszemy krótkie streszczenie pracy w języku angielskim (nie powinno być dłuższe niż 530 znaków).}
%
% Podajemy najważniejsze słowa kluczowe po polsku i angielsku
% - w obu przypadkach, nie więcej niż 150 znaków.
\slowakluczowe{tutaj podajemy najważniejsze słowa kluczowe (łącznie nie powinny być dłuższe niż 150 znaków).}  
\slowakluczoweang{tutaj podajemy najważniejsze słowa kluczowe w języku angielskim (łącznie nie powinny być dłuższe niż 150 znaków)}
%
%
%%%%%%%%%%%%%%%%%%%%%%%%%%%%%%%%%%%%%%%%%%%%%%%%%%%%%%%%%
% Definicje, lematy, twierdzenia, przykłady i wnioski
% Komendy wywołujące twierdzenia, definicje, itd., 
% czyli 'theorem', 'definition', 'corollary', itd., 
% można zmienić wedle uznania.
\theoremstyle{plain}
\newtheorem{theorem}{Twierdzenie}
\numberwithin{theorem}{chapter}
\newtheorem{lemma}[theorem]{Lemat} 
\newtheorem{corollary}[theorem]{Wniosek}
\newtheorem{fact}[theorem]{Fakt}
\theoremstyle{definition}
\numberwithin{theorem}{chapter}
\newtheorem{definition}[theorem]{Definicja} 
\newtheorem{example}[theorem]{Przykład}
\newtheorem{note}[theorem]{Uwaga}
%%%%%%%%%%%%%%%%%%%%%%%%%%%%%%%%%%%%%%%%%%%%%%%%%%%%%%%%%


%%%%%%%%%%%%%%%%%%%%%%%%%%%%%%%%%%%%%%%%%%%%%%%%%%%%%%%%%
%%%%%%%%%%%%%%%%%%%%%%%%%%%%%%%%%%%%%%%%%%%%%%%%%%%%%%%%%
\begin{document}
\frontmatter
\maketitle
\mainmatter
\tableofcontents
%\listoffigures
%\listoftables

{\backmatter \chapter{Wstęp}}
We wstępie zapowiadamy, o czym będzie praca. Próbujemy zachęcić czytelnika do dalszej lektury, np. krótko informując, dlaczego wybraliśmy właśnie ten temat i co nas w nim zainteresowało.

%\chapter{Rozdział pierwszy}
%Tabela \ref{tab:przykladowa} przedstawia przykładową tabelę. Do tworzenia tabeli służą m.in. środowiska \texttt{tabular} oraz \texttt{table}. Istnieje możliwość numeracji dwustopniowej, gdzie pierwsza cyfra oznacza numer rozdziału, a druga – kolejny numer tabeli w tym rozdziale. Tytuł powinien znajdować się centralnie nad tabelą, $12$ pkt odstępu od tekstu zasadniczego nad i pod tabelą wraz z tytułem. Jeśli tabela jest cytowana – należy podać centralnie pod tabelą źródło jej pochodzenia, np. opracowanie własne, opracowano na podstawie danych z GUS.
%\begin{table}[ht]
%\caption{Podstawowa Tabela}
%\centering
%\begin{tabular}{ccc}
%\hline
%\hline                       
%Państwo & PKB (w milionach USD )& Stopa bezrobocia  \\  [0.5ex] 
%\hline 
%Stany Zjednoczone & 75 278 049 & 4,60\%  \\
%Chiny & 11 218 281 & 4,10\%   \\
%Japonia & 4 938 644 & 3,10\%  \\
%Niemcy & 3 466 639 & 6,00\%   \\
%Wielka Brytania & 2 629 188 & 4,60\%  \\ [1ex]  
%\hline 
%\end{tabular}
%\caption*{\textit{Źródło: opracowanie własne}}
%\label{tab:przykladowa} 
%\end{table}
%
%Do cytowania używamy komendy \texttt{cite}. W nawiasie klamrowym podajemy klucz, którego użyliśmy w pliku \emph{bibliografia.bib}. Przykład: \cite{einstein} lub \cite[chap. 2]{latexcompanion}.
%
%\section{Podrozdział pierwszy}
%
%\begin{table}[H]
%\caption{Podstawowa Tabela}
%\centering
%\begin{tabular}{ccc}
%\hline
%\hline                       
%Państwo & PKB (w milionach USD )& Stopa bezrobocia  \\  [0.5ex] 
%\hline 
%Stany Zjednoczone & 75 278 049 & 4,60\%  \\
%Chiny & 11 218 281 & 4,10\%   \\
%Japonia & 4 938 644 & 3,10\%  \\
%Niemcy & 3 466 639 & 6,00\%   \\
%Wielka Brytania & 2 629 188 & 4,60\%  \\ [1ex]  
%\hline 
%\end{tabular}
%\caption*{\textit{Źródło: opracowanie własne}}
%\label{tab:przykladowa2} 
%\end{table}
%
%\section{Podrozdział drugi}
%
%Rysunki do pracy dyplomowej należy wstawiać w sposób podobny do wstawiania tabel, z~zasadniczą różnicą polegającą na tym, że podpis powinno umieszczać się centralnie pod rysunkiem, a nie powyżej niego. Numeracja i sposób cytowania pozostają bez zmian, przy czym tabele i rysunki nie mają numeracji wspólnej, np. po Tabeli \ref{tab:przykladowa2} występuje Rysunek \ref{rys1} (o ile jest to pierwszy rysunek rozdziału pierwszego), a nie Rysunek $1.3$.
%
%\begin{figure}[ht]
%
%\centering
%                     
%\includegraphics[scale=0.27]{logo_w13.jpg}
%\caption{Podstawowy Rysunek}\label{rys1}
%\end{figure}
%\label{rys:przykladowy} 
%
%
%
%\chapter{Definicje, lematy, twierdzenia, przykłady i wnioski}
%Definicje, lematy, twierdzenia, przykłady i wnioski piszemy w pracy tak:
%\begin{definition}[Martyngał]
%Tu piszemy treść definicji martyngału.
%\end{definition}
%\begin{lemma}[]% w nawiasie kwadratowym można napisać jego nazwę
%Tu piszemy treść lematu.
%\end{lemma}


% MOJE ================================================
\chapter{Definicje, lematy, twierdzenia, przykłady i wnioski}

\section{Definicje}

\subsection{Definicje ogólne}

\begin{enumerate}
\item wnetrze zbioru
\item topologia
\item baza topologii
\item przestrzeń topologiczna
\item metryka
\item przestrzeń metryczna
\item kula otwarta
\item gęstość w przestrzeni metrycznej
\item przestrzen metryczna zwarta,
\item punkt izolowany w przestrzeni metrycznej


\item Ciaglosc funkcji na przestrzeni metrycznej (zbior $C(X)$)


\item uklad dynamiczny,
\item orbita,
\item orbita okresowa
\item punkt okresowy odwzorowania
\item niezmienniczosc zbioru ze wzgledu na odwzorowanie

\item topologiczna tranzytywnosc
\item entropia topologiczna
\item RODZAJE CHAOSU (Devaneya, Li Yorka)


\item napisac ze chaotycznosc odwzorowania rozumiem przez chaotycznosc odpowiedniego ukladu dynamicznego

\item zbiór rezydualny
\item separable, second category (czyli 1 i 2 kategoria bairea)
TODO: separable znaczy chyba osrodkowa??
\item g-delta




\item odwzorowania trójkątne, zbior $C_\triangle(X \times I)$

\end{enumerate}

% Przyklad cytowania z numerem strony
%\begin{definition}[Metryka]
%TODO
%\cite[s.~103]{kuratowski1977wstep}
%\end{definition}

\begin{definition}[Metryka]
Metryką na zbiorze $X$ nazywamy funkcję $\rho : X \times X \longrightarrow \mathbb{R}_+ \cup \{0\}$ spełniającą następujące warunki:
\begin{enumerate}
\item $\forall_{x,y \in X}: \rho(x,y)=0 \iff x=y$,
\item $\forall_{x,y \in X}: \rho(x,y) = \rho(y,x)$,
\item \label{nierownosc_trojkata} $\forall_{x,y,z \in X}: \rho(x,y) \leq \rho(x,z) + \rho(z,y)$.
\end{enumerate}
Warunek \ref{nierownosc_trojkata} nazywany jest zwykle nierównością trójkąta.

\end{definition}

\begin{definition}[Przestrzeń metryczna]
Przestrzenią metryczną nazywamy parę $(X, \rho)$, gdzie $X$ jest zbiorem a $\rho$ zdefiniowaną na nim metryką.
\end{definition}

\begin{definition}[Kula otwarta]
Kulą otwartą w przestrzeni metrycznej $(X, \rho)$ nazywamy zbiór: $K(s, r) = \{x \in X: \rho(s, x) < r\}$.
Punkt $s$ nazywamy wówczas środkiem kuli $K$, a $r \in \mathbb{R}_+ \cup \{0\}$ jej promieniem.
\end{definition}

\begin{definition}[Zbiór gęsty]
Dla danej przestrzeni metrycznej $(X, \rho)$. Zbiór $A \subset X$ nazwiemy gęstym, gdy $\forall_{x \in X} \, \forall_{\epsilon>0} \, \exists_{a \in A} : \rho(x, a) < \epsilon$.
\end{definition}



\subsection{Definicje utworzone na potrzeby dowodu twierdzenia o rozszerzaniu}

Na potrzeby dowodu wprowadźmy pojęcia odległości między odwzorowaniami oraz dwie funkcje: $\textrm{pr}_1(x, y)$ i $\textrm{pr}_2(x, y)$. 

\begin{definition}[Metryka na przestrzeni funkcji ciągłych w przestrzeni metrycznej]
Niech $(M, \sigma)$ będzie zwartą przestrzenią metryczną, rozważmy odwzorowania $h,k \in C(M)$. Odległość między nimi zdefiniujmy jako $\max_{m \in M} \sigma(h(m), k(m))$ i oznaczmy ją jako $d_1(h,k)$.
\end{definition}

\begin{definition}[Metryka na przestrzeni odwzorowań trójkątnych]
Odległość między odwzorowaniami trójkątnymi definiujemy wówczas następująco: Niech $(X, \rho)$ i $(Y, \tau)$ będą zwartymi przestrzeniami metrycznymi a $F(x,y) = (f(x), g_x(y))$ i $\Phi(x,y) = (\phi(x), \psi_x(y))$ trójkątnymi odwzorowaniami należącymi do $C_\triangle(X \times Y)$. Odległość definiujemy wówczas jako 
\begin{align*}
d_2(F, \Phi) = \max_{(x,y \in X \times Y} \max\{\rho(f(x),\phi(x)), \tau(g_x(y), \psi_x(y))\} \\ 
= \max\{d_1(f,\phi), \max_{x \in X}d_1(g_x, \psi_x)\}
\end{align*}
Zauważmy, że jak wynika z lematu \ref{przestrzenie_ciaglych_i_ciaglych_trojkatnych_sa_zupelne} przestrzenie metryczne $(C(X), d_1)$ oraz $(C_\triangle(X \times Y), d_2)$ są zupełne \textcolor{red}{ i odpowiednie topologie na nich są topologiami jednostajnej zbieżności. (TODO czy to jest właściwe tłumaczenie?) } 
\end{definition}

\begin{definition}[$\textrm{pr}_1(x, y)$, $\textrm{pr}_2(x, y)$]
Dla $(x, y) \in X \times Y$ niech $\textrm{pr}_1(x, y) = x$ i $\textrm{pr}_2(x, y) = y$. Odwzorowanie identycznościowe na $Y$ będziemy oznaczać przez $\textrm{Id}_Y$ lub krótko $\textrm{Id}$. W dalszej części pracy przestrzeń $Y$ będzie odcinkiem rzeczywistym $I = [0, 1]$.
\end{definition}





\section{Lematy}


\begin{lemma}\label{przestrzenie_ciaglych_i_ciaglych_trojkatnych_sa_zupelne}
% TODO
\end{lemma}

\begin{lemma}\label{nigdziegestosc_orbit}
Niech  $(X, \rho)$ będzie zwartą przestrzenią metryczną bez punktów izolowanych. Każda okresowa orbita $P_0$ odwzorowania $f \in C(X)$ jest nigdziegęstym domkniętym podzbiorem $X$.
\end{lemma}

\begin{lemma}\label{rownosc_entropii_gdy_wlokna_monotoniczne}
% TODO
\end{lemma}

\begin{lemma}
\label{warunki_dostateczne_chaotycznosci_devaneya}
% TODO
dla przestrzeni nieskonczonej i zwartej czyli referencja 25 z pracy glownej
\end{lemma}

\begin{lemma} \label{F_jest_niepustym_domknietym_podzbiorem_trojkatnych}
% TODO
\end{lemma}

\begin{lemma} \label{trojkatne_tworza_przestrzen_metryczna_zupelna}
% TODO
\end{lemma}

\begin{lemma} \label{przekroj_rezydualnych_jest_rezydualny}
% TODO
\end{lemma}

\begin{theorem} \label{tranzytywne_rezydualne_w_F}
% TODO
\cite{alseda1999entropy} - dowod twierdzenia 1.5
\end{theorem}

\begin{lemma} \label{Fso_sa_otwartymi_podzbiorami_F}
% TODO
\end{lemma}

\begin{lemma}
\label{w_otoczeniu_niezmienniczego_jest_otwarty_zawarty_w_iteracjach}
% TODO
\end{lemma}

\begin{lemma}
\label{lemat_3_glownego_artykulu_istnieje_orbita_okresowa_krojaca_sie_z_rodzina_otwartych}
% TODO
W glownej pracy to byl lemat 3
\end{lemma}

\begin{lemma}
\label{wszystkie_wloknowe_na_implikuje_punkt_okresowy}
% TODO
\end{lemma}

\begin{lemma}
\label{lemat_4_glownego_artykulu}
\cite{alseda1999entropy}
Niech $(X, \rho)$ będzie zwartą przestrzenią metryczną i niech $F = (f, g_x)$ będzie odwzorowaniem należącym do $C_\triangle(X \times I)$ którego wszystkie włókna są niemalejące i pozostawiają krańce $I$ nie zmienione. Niech $\{a_1, a_2, \ldots, a_n\}$ będzie podzbiorem $X$ oraz dla $i = 1, 2, \ldots, n$ niech $U_i$ będą parami rozłącznymi zbiorami otwartymi takimi, że $a_i \in U_i$. Załóżmy, że $h_i$ są niemalejącymi odwzorowaniami z $C(I)$ pozostawiającymi krańce $I$ niezmienione i spełniającymi $d_1(h_i, g_{a_i}) < \epsilon$ dla pewnego dodatniego $\epsilon$ i każdego $i = 1, 2, \ldots, n$. Wówczas istnieje odwzorowanie $\widetilde{F} = (f, \widetilde{g}_x) \in C_\triangle(X \times I)$ spełniające cztery następujące warunki:

\begin{enumerate}
\item  wszystkie włókna $\widetilde{F}$ są niemalejące i pozostawiające krańce $I$ niezmienione,
\item $d_2(F, \widetilde{F}) < \epsilon$,
\item $\widetilde{g}_{a_i} = h_i$ dla $i = 1,2,\ldots,n$,
\item $\widetilde{g}_x = g_x$ dla $x \in X \setminus \bigcup_{i=1}^n U_i$.
\end{enumerate}

\begin{proof}
\cite{alseda1999entropy}
Dla każdego $i = 1, 2, \ldots, n$ niech $V_i \subset U_i$ będzie otwartym sąsiedztwem $a_i$ takim, że dla pewnego dodatniego $\widetilde{\epsilon} < \epsilon$, $d_1(h_i, g_x) < \widetilde{\epsilon}$ zawsze wtedy gdy $x \in V_i$.

Oznaczmy $U = \bigcup_{i=1}^n U_i$, $V = \bigcup_{i=1}^n V_i$ i weźmy $u : X \longrightarrow [0,1]$, ciągłą funkcję, przyjmującą wartość $1$ na zbiorze $\{a_1, a_2, \ldots, a_n\}$, natomiast $0$ poza zbiorem $V$. Zastąpmy każde odwzorowanie włóknowe $g_x$ przez $\widetilde{g}_x$, gdzie

\begin{equation} \label{lemat4_rownanie_1}
    \widetilde{g}_x(y) =
    \begin{cases}
        g_x(y) & \text{if $x \in X \setminus V$,}\\
        g_x(y)(1-u(x))+h_i(y)u(x) & \text{if $x \in \{1,2,\ldots,n\}$.}
    \end{cases}
\end{equation}

dla każdego $y \in I$. Zauważmy ponadto, że dla $x \in V_i$ oraz $i \in \{1,2,\ldots,n\}$ możemy równoważnie napisać

\begin{equation} \label{lemat4_rownanie_2}
\widetilde{g}_x(y) = u(x)(h_i(y) - g_x(y)) + g_x(y).
\end{equation}

Rozważmy odwzorowanie $\widetilde{F} = (f, \widetilde{g}_x)$. Należy ono do $C_\triangle(X \times I)$. Z równości \ref{lemat4_rownanie_1} widzimy, że wszystkie włókna $\widetilde{g}_x$ są niemalejące i pozostawjaiąc krańce przedziału $I$ nie zmienione (TODO sprawdzic czy to dobre tluamczenie, mzoe krance ustalone? krance stale?). Ponadto $\widetilde{g}_{a_i} = h_i$ dla każdego $i$ oraz $\widetilde{g}_x = g_x$ dla $x \in X \setminus V \supset X \setminus U$. Ponieważ dla $x \in V_i$, gdzie $i \in \{1,2,\ldots,n\}$ mamy $d_1(h_i, g_x) < \widetilde{\epsilon}$ oraz $u(x) \in [0,1]$, zatem z równości \ref{lemat4_rownanie_2} otrzymujemy $d_1(g_x, \widetilde{g}_x) < \widetilde{\epsilon}$ dla każdego $x \in V$. Wynika z tego, że $d_2(F, \widetilde(F)) \leq \widetilde{\epsilon} < \epsilon$ (TODO uzasadnic tutaj wewnatrz to ostatnie przejscie od d1 do d2 z definicji d1 i d2), co kończy dowód.

\end{proof}



\end{lemma}

\begin{lemma}
\label{przeliczalny_przekroj_gestych_jest_rezydualny} 
% TODO
\end{lemma}

\begin{lemma}
\label{nadzbior_rezydualnego_jest_rezydualny}
% TODO
\end{lemma}

\begin{lemma}
\label{uzasadnienie_rownosci_G_z_potegami}
% TODO
\end{lemma}

\begin{lemma}
\label{obraz_x0_razy_ab_przez_iterate_G_rowna_sie_x0y0}
% TODO
\end{lemma}

\begin{lemma}
\label{obraz_przez_iterate_kazdego_dostatecznie_bliskiego_G_zawiera_sie_w_x0_razy_ab}
% TODO
\end{lemma}

\begin{lemma}
\label{na_mocy_lematu_4_mozemy_dostac_odwzorowanie_H_z_wloknami_scisle_rosnacymi}
% TODO
\end{lemma}

\begin{lemma}
\label{y0_rozne_od_0_i_1_implikuje_pr2Hrx0y0_rozne_od_0_i_1}
% TODO
\end{lemma}

\begin{lemma}
\label{przypadek1_daje_odwzorowanie_G_bliskie_H}
% TODO
\end{lemma}





\chapter{Twierdzenie o rozszerzaniu odwzorowań chaotycznych w sensie Devaney'a}
\begin{theorem}[O rozszerzaniu]\label{twierdzenie_glowne}
Niech $(X, \rho)$ będzie zwartą przestrzenią metryczną bez punktów izolowanych oraz niech $f \in C(X)$ będzie odwzorowaniem chaotycznym w sensie Devaney'a. Wówczas odwzorowanie $f$ można rozszerzyć do odwzorowania $F \in C_{\triangle}(X \times I)$ (to znaczy tak, że $f$ jest odwzorowaniem bazowym dla $F$) w taki sposób, że:
\begin{enumerate}[label=(\roman*)]
\item\label{twierdzenie_glowne_a} $F$ jest również chaotyczne w sensie Devaney'a, 
\item\label{twierdzenie_glowne_b} $F$ ma taką samą entropię topologiczną jak $f$, 
\item\label{twierdzenie_glowne_c} zbiory $X \times \{0\}$ i $X \times \{1\}$ są niezmiennicze ze względu na $F$.
\end{enumerate}
\end{theorem}
\cite{balibrea2003topological}


\begin{proof}[Dowód twierdzenia o rozszerzaniu]
Odwzorowanie $f$ jest chaotyczne w sensie Devaney'a, zatem spełnia warunek $(2)$, czyli ma gęsty zbiór punktów okresowych, w szczególności istnieje orbita okresowa.
Możemy zatem ustalić okresową orbitę $P_0$ odwzorowania $f$. Z lematu \ref{nigdziegestosc_orbit} mamy, że $P_0$ jest nigdziegęstym, domkniętym podzbiorem $X$.

Rozważmy zbiór $\mathcal{F}$ wszystkich odwzorowań $F = (f, g_x)$ ze zbioru $C_\triangle(X \times I)$ spełniających następujące warunki:
\begin{enumerate}
\item\label{dowod_glowny_a} Odwzorowanie bazowe $f$ spełnia założenia twierdzenia \ref{twierdzenie_glowne}.
\item\label{dowod_glowny_b} $\forall_{x \in X}$ odwzorowanie $g_x$ jest niemalejące i krańce przedziału I pozostawia niezmienione.
\item\label{dowod_glowny_c} $\forall_{x \in P_0}$ $g_x$ jest identycznością 
\end{enumerate}

Warunek \ref{dowod_glowny1} implikuje, że dla każdego odwzorowania z $\mathcal{F}$ zbiory $X \times \{0\}$ i $X \times \{1\}$ są niezmiennicze, czyli $\forall_{F \in \mathcal{F}}$ zachodzi warunek \ref{twierdzenie_glowne_c}  twierdzenia \ref{twierdzenie_glowne}.
Zachodzenie warunku \ref{twierdzenie_glowne_b} twierdzenia \ref{twierdzenie_glowne} dla każdego odwzorowania $F \in \mathcal{F}$ wynika z lematu \ref{rownosc_entropii_gdy_wlokna_monotoniczne}.
Pozostaje zatem wykazać prawdziwość warunku \ref{twierdzenie_glowne_a}, czyli chaotyczność w sensie Devaney'a jakiegoś odwzorowania $F \in \mathcal{F}$. Takie odwzorowanie będzie bowiem łącznie spełniało wszystkie 3 warunki, czyli tezę twierdzenia.

Z lematu \ref{warunki_dostateczne_chaotycznosci_devaneya} wynika, że aby odwzorowanie $F$ było chaotyczne w sense Devaneya potrzeba i wystarcza, żeby spełniało dwa poniższe warunki:
\begin{enumerate}
\item \label{devaney_pierwsza_wlasnosc} $F$ jest topologicznie tranzytywne
\item \label{devaney_druga_wlasnosc} zbiór punktów okresowych odwzorowania $F$ jest gęsty w $(X \times I)$
\end{enumerate}
Jest tak gdyż przestrzeń $(X \times I)$ spełnia założenia lematu, tj. jest przestrzenią nieskończoną i zwartą.

Chcemy wykazać, że $\exists_{F \in \mathcal{F}}$ będące jednocześnie topologicznie tranzytywne i posiadające gęsty w $(X \times I)$ zbiór punktów okresowych. Z lematu \ref{F_jest_niepustym_domknietym_podzbiorem_trojkatnych}wiemy, że$\mathcal{F}$ jest niepustym, domkniętym podzbiorem ptrzestrzeni $C_{\triangle}(X \times I)$, która jak wynika z lematu \ref{trojkatne_tworza_przestrzen_metryczna_zupelna} jest przestrzenią metryczną zupełną. 

Przekrój dwóch zbiorów rezydualnych jest rezydualny (patrz lemat \ref{przekroj_rezydualnych_jest_rezydualny}) a więc niepusty. Wystarczy zatem pokazać, że oba zbiory:
\begin{itemize}
\item zbiór odwzorowań topologicznie tranzytywnych
\item zbiór odwzorowań, których zbiór punktów okresowych jest gęsty w $(X \times I)$
\end{itemize}
są rezydualne w $\mathcal{F}$. Wówczas każde odwzorowanie należące do ich przekroju będzie spełniało wszystkie trzy warunki tezy twierdzenia \ref{twierdzenie_glowne} 

Zbiór odwzorowań tranzytywnych jest rezydualny w $\mathcal{F}$, zostało to udowodnione w twierdzeniu \ref{tranzytywne_rezydualne_w_F}

Pozostało wykazać, że również zbiór odwzorowań posiadających gęsty zbiór punktów okresowych jest rezydualny w $\mathcal{F}$.
Oznaczmy zbiór takich odwzorowań (jednocześnie należących do $\mathcal{F}$) przez $\mathcal{F}_{DP}$.

Niech $\{U_i^X\}_{i=1}^{\infty}$ będzie bazą topologii $X$ i niech $\{U_i^I\}_{i=1}^{\infty}$ będzie zbiorem wszystkich odcinków otwartych o końcach wymiernych, należących do odcinka otwartego $(0, 1)$. Niech $\{U_i\}_{i=1}^{\infty}$ będzie ponumerowaniem zbioru $\{U_i^X \times U_j^I : i,j \in \mathbb{N}\}$. Wtedy każda kula otwarta w $X \times I$ zawiera jakiś spośród otwartych zbiorów $U_i$.

Dla każdego $i=1,2,...$ niech zbiór $\mathcal{F}_{SO}^i$ ("S" - stabilny, "O" - okresowy) będzie zdefiniowany następująco. Odwzorowanie G należy do $\mathcal{F}_{SO}^i$ wtedy i tylko wtedy, gdy należy do $\mathcal{F}$, posiada punkt okresowy w $U_i$ oraz wszystkie dostatecznie bliskie $G$ odwzorowania z $\mathcal{F}$ również posiadają punkt okresowy w $U_i$ (być może różne od punktuów okresowych odwzorowania $G$). Zbiory $\mathcal{F}_{SO}^i$ są otwartymi podzbiorami $\mathcal{F}$ (patrz lemat \ref{Fso_sa_otwartymi_podzbiorami_F}). Ponieważ $\mathcal{F}_{DP} \supseteq \bigcap_{i=1}^{\infty} \mathcal{F}_{SO}^i$ aby pokazać, że $\mathcal{F}_{DP}$ jest rezydualny w $\mathcal{F}$ wystarczy pokazać, że $\forall_{i \in \mathbb{N}} \mathcal{F}_{SO}^i$ jest gęsty w $\mathcal{F}$. (Wynika to z faktów: \ref{przeliczalny_przekroj_gestych_jest_rezydualny}  i \ref{nadzbior_rezydualnego_jest_rezydualny}).

Aby wykazać że każdy zbiór $\mathcal{F}_{SO}^i$ jest gęsty w $\mathcal{F}$ ustalmy dowolne: $i \in \mathbb{N}$, $F=(f,g_x) \in \mathcal{F}$ i $\epsilon > 0$. Pokażemy, że istnieje odwrorowanie $G \in \mathcal{F}_{SO}^i$, którego odległość od $F$ nie przekracza $\epsilon$. Dla uproszczenia sytuacji załóżmy, że $\rho(\textrm{pr}_1(U_i), P_0) > 0$ (Jeżeli tak nie jest, zawsze możemy wziąć zamiast $U_i$ mniejszy prostokąt $U_i^* \subset U_i$).

Weźmy dodatnią liczbę naturalną $N \geq \frac{4}{\epsilon}$. Następnie rozważmy otwarte sąsiedztwo $V$ orbity $P_0$ w przestrzeni $(X, \rho)$ takie, że $\rho(\textrm{pr}_1(U_i), V) \geq 0$ oraz $d_1(g_x, \textrm{Id}) < \frac{\epsilon}{4}$ dla każdego $x \in V$ (pamiętamy, że $g_x = \textrm{Id}$ dla $x \in P_0$).

$P_0$ jest zbiorem niezmienniczym ze względu na f, na mocy lematu \ref{w_otoczeniu_niezmienniczego_jest_otwarty_zawarty_w_iteracjach} istnieje niepusty zbiór otwarty $W \subseteq V$ taki, że $W \cup f(W) \cup \ldots \cup f^N(W) \subseteq V$. Na mocy lematu \ref{lemat_3_glownego_artykulu_istnieje_orbita_okresowa_krojaca_sie_z_rodzina_otwartych} istnieje punkt okresowy $x_0$ odwzorowania $f$ taki, że $x_0 \in \textrm{pr}_1(U_i)$ oraz orbita $x_0$ kroi się niepusto ze zbiorem $W$. Niech $r > 0$ będzie pierwszą dodatnią liczbą całkowitą dla której $f^r(x_0) \in W$. Wtedy $f^r(x_0), f^{r+1}(x_0), \ldots, f^{r+N-1}(x_0) \in V$. Niech $s \geq 0$ będzie pierwszą nieujemną liczbą całkowitą dla której $f^{r+N+s}(x_0) = x_0$, tj. $r+N+s$ jest okresem punktu $x_0$. Weźmy $y_0$ takie, że $(x_0, y_0) \in U_i$. Ponieważ wszystkie odwzorowania włóknowe ($g_x$) odwzorowania $F$ są "na", to istnieje punkt $y^* \in (0,1)$ taki, że $F^s(f^{r+N}(x_0), y^*) = (x_0, y_0)$ (patrz lemat \ref{wszystkie_wloknowe_na_implikuje_punkt_okresowy}).

Przypadek 1. $z = \textrm{pr}_2(F^r(x_0, y_0)$ jest różne od 0 i 1.
Oznaczmy przez $g$ odwzorowanie z $C(I)$ posiadające następujące trzy własności:
\begin{enumerate}[label=(g\arabic*)]
\item \label{g_jeden} $d_1(g, \mathrm{Id}) < \frac{\epsilon}{4}$,
\item \label{g_dwa} $g$ jest odwzorowaniem niemalejącym, pozostawiającym końce przedziału $I$ niezmienione,
\item \label{g_trzy} $g^N(z) = y^*$.
\end{enumerate}

Następnie, rozważmy odwzorowanie $h \in C(I)$ posiadające trzy następujące własności:
\begin{enumerate}[label=(h\arabic*)]
\item \label{h_jeden} $d_1(h, g_{x_0}) < \frac{\epsilon}{4}$,
\item \label{h_dwa} $h(y_0) = g_{x_0}(y_0)$,
\item \label{h_trzy} $h$ jest stałe na zwartym odcinku $[a,b] \subseteq \textrm{pr}_2(U_i)$ zawierającym punkt $y_0$ w swoim wnętrzu.
\end{enumerate}

Weźmy teraz odwzorowanie $G = (f, \widetilde{g}_x) \in \mathcal{F}$ takie, że $d_2(G, F) < \frac{\epsilon}{2}$ oraz

\begin{equation}
    \widetilde{g}_x =
    \begin{cases}
        h & \text{if $x=x_0$,}\\
        g & \text{if $x \in \{f^k(x_0) : r \leq k \leq r+N-1\}$,}\\
        g_x & \text{if $x \in \{f^k(x_0) : 1 \leq k \leq r-1$ or $r+N \leq k \leq r+N+s-1\}$.}
    \end{cases}
\end{equation}

Odwzorowanie takie istnieje na mocy lematu \ref{lemat_4_glownego_artykulu}, ponieważ
$$d_1(\widetilde{g}_{x_0}, g_{x_0}) = d_1(h, g_{x_0}) < \frac{\epsilon}{4}$$
oraz dla $x \in \{f^r(x_0), f^{r+1}(x_0), \ldots, f^{r+N-1}(x_0)\}$,
$$d_1(\widetilde{g}_x, g_x) = d_1(g, g_x) \leq d_1(g, \mathrm{Id}) + d_1(\mathrm{Id}, g_x) < \frac{\epsilon}{4} + \frac{\epsilon}{4} = \frac{\epsilon}{2}$$.

Punkt $(x_0, y_0) \in U_i$ jest punktem okresowym odwzorowania $G$, ponieważ

\begin{equation}
\begin{split}
G^{r+N+s}(x_0, y_0) & = G^{r+N+s-1}(G(x_0, y_0)) \\
& = G^{r+N+s-1}(F(x_0, y_0)) = G^{N+s}(F^r(x_0, y_0)) \\
& = G^{N+s}(f^r(x_0), z) = G^s(f^{r+N}(x_0), y^*) \\
& = F^s(f^{r+N}(x_0), y^*) = (x_0, y_0).
\end{split}
\end{equation}
Szczegółowe uzasadnienie powyższej równości znajduje się w \ref{uzasadnienie_rownosci_G_z_potegami}

Ponadto, ze względu na \ref{h_trzy}, na mocy \ref{obraz_x0_razy_ab_przez_iterate_G_rowna_sie_x0y0} zachodzi
$$G^{r+N+s}(\{x_0\} \times [a, b]) = \{(x_0, y_0)\}.$$

Z faktu $y_0 \in (a, b)$ oraz \ref{obraz_przez_iterate_kazdego_dostatecznie_bliskiego_G_zawiera_sie_w_x0_razy_ab} wynika, że każde odwzorowanie $\widetilde{G} \in \mathcal{F}$ dostatecznie bliskie $G$ posiada własność
$$\widetilde{G}^{r+N+s}(\{x_0\} \times [a, b]) \subseteq \{x_0\} \times [a, b].$$


Zatem $\widetilde{G}$ posiada punkt okresowy w $\{x_0\} \times [a, b] \subseteq U_i$. Zatem $G \in \mathcal{F}_{SO}^i$, co kończy dowód dla przypadku 1.

Przypadek 2. $\textrm{pr}_2(R^r(x_0, y_0))$ jest równy $0$ lub $1$.

W takim przypadku użyjemy lematu \ref{lemat_4_glownego_artykulu} aby dostać odwzorowanie $H = (f, h_x) \in \mathcal{F}$ takie, że $d_2(H, F) < \frac{\epsilon}{2}$ i dla $x \in \{x_0, f(x_0), \ldots, f^{r-1}(x_0)\}$ odwzorowania włóknowe $h_x$ są ściśle rosnące. (Odwzorowanie takie istnieje na mocy \ref{na_mocy_lematu_4_mozemy_dostac_odwzorowanie_H_z_wloknami_scisle_rosnacymi}) Ponieważ $y_0$ jest różnie od $0$ i $1$ dostajemy, że $\textrm{pr}_2(H^r(x_0, y_0))$ również jest różne od $0$ i $1$ (uzasadnienie w \ref{y0_rozne_od_0_i_1_implikuje_pr2Hrx0y0_rozne_od_0_i_1}). Następnie korzystając z przypadku 1 i z lematu \ref{przypadek1_daje_odwzorowanie_G_bliskie_H} dostajemy odwzorowanie $G \in \mathcal{F}_{SO}^i$, dla którego zachodzi nierówność $d_2(G, H) < \frac{\epsilon}{2}$. Wówczas $d_2(G, F) < \epsilon$. (TODO powolac sie na nierownosc trojkata)


\end{proof}




% END MOJE ====================================================



{\backmatter \chapter{Podsumowanie}}
Podsumowanie w pracach matematycznych nie jest obligatoryjne. Warto jednak na zakończenie krótko napisać, co udało nam się zrobić w pracy, a czasem także o tym, czego nie udało się zrobić.

{\backmatter \chapter{Dodatek}}
Dodatek w pracach matematycznych również nie jest wymagany. Można w nim przedstawić np. jakiś dłuższy dowód, który z pewnych przyczyn pominęliśmy we właściwej części pracy lub (np. w przypadku prac statystycznych) umieścić dane, które analizowaliśmy.

%%%%%%%%%%%%%%%%%%%%%%%%%%%%%%%%%%%%%%%%%%%%%%%%%%%%%%%%%
% BIBLIOGRAFIA
% W tworzeniu bibliografii najlepiej korzystać z BibTex'a, 
% który jest częścią systemu Tex. W naszym przypadku funkcję 
% przechowalni literatury, do której się odwołujemy, pełni 
% plik bibliografia.bib. Nie musimy ręcznie dodawać nowych 
% pozycji do bibliografii. Możemy wejść np. na stronę 
% https://mathscinet.ams.org/mathscinet/index.html, 
% znaleźć odpowiednią pozycję, wybrać ją, a następnie zmienić 
% 'Select alternative format' na BibTeX, skopiować uzyskany 
% tekst, wkleić do pliku bibliografia.bib i skompilować. 
% Gotowe informacje do pliku bibliografia.bib można znaleźć 
% także na https://arxiv.org - gdy znajdziemy interesującą nas 
% pracę, szukamy 'References & Citations' i klikamy 'NASA ADS', 
% a potem 'Bibtex entry for this abstract' 
% i postępujemy tak jak wcześniej.
%%%%%%%%%%%%%%%%%%%%%%%%%%%%%%%%%%%%%%%%%%%%%%%%%%%%%%%%%
\newpage
% w nawiasie klamrowym wpisujemy nazwę pliku z bibliografią w formacie .bib
\bibliografia{bibliografia.bib} 
\end{document}