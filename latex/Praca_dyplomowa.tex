%%%%%%%%%%%%%%%%%%%%%%%%%%%%%%%%%%%%%%%%%%%%%%%%%%%%%%%%%
% Niniejszy plik przedstawia przykładowy skład 
% pracy dyplomowej na Wydziale Matematyki PWr. 
% 
% Autorzy: 
% Damian Fafuła
% Michał Kijaczko
% Jakub Michalczak
% Maciej Miśta
% Dagmara Nowak
% Tomasz Skalski
% Wojciech Słomian
%
%% Data utworzenia: 8.05.2018
% Numer wersji: 1
%
% Poniższą formatkę można rozpowszechniać i edytować 
% pod warunkiem zachowania numeru wersji, 
% informacji o autorach i dodaniu informacji 
% o wprowadzonych zmianach.
%
%%%%%%%%%%%%%%%%%%%%%%%%%%%%%%%%%%%%%%%%%%%%%%%%%%%%%%%%%
% Domyślną opcją jest: praca magisterska, język polski.
% W przypadku pracy pisanej w języku angielskim dodajemy 
% opcję [english].
% Dla pracy licencjackiej dodajemy opcję [licencjacka].
% Dla pracy inżynierskiej dodajemy opcję [inzynierska].
% Dopuszczalne są podwójne opcje, np. [licencjacka, english].
% Opcje dodajemy w kwadratowym nawiasie przy \documentclass.
%
%
%%%%%%%%%%%%%%%%%%%%%%%%%%%%%%%%%%%%%%%%%%%%%%%%%%%%%%%%%
\documentclass[licencjacka]{pwr_wmat_praca_dyplomowa}

\usepackage{enumitem}
%%%%%%%%%%%%%%%%%%%%%%%%%%%%%%%%%%%%%%%%%%%%%%%%%%%%%%%%%
%              DANE DO PRACY
%
% W przypadku pracy dyplomowej w języku angielskim nie jest konieczne 
% wypełnianie pól: \tytul{}, \kierunek{}, \specjalnosc{}, 
%                  \streszczenie{}, \slowakluczowe{}.
%%%%%%%%%%%%%%%%%%%%%%%%%%%%%%%%%%%%%%%%%%%%%%%%%%%%%%%%%
%
% Imię i nazwisko autora
\autor{Imię i nazwisko dyplomanta}
%
% Tytuł pracy dyplomowej 
\tytul{Tytuł pracy dyplomowej} 
\tytulang{Tytuł pracy dyplomowej w języku angielskim}
%
% Tytuł / stopień / imię i nazwisko opiekuna
\opiekun{dr inż. Dawid Huczek}
%
% Kierunek studiów wybieramy spośród następujących:
% 1) Matematyka
% 2) Matematyka i Statystyka
% 3) Matematyka stosowana
\kierunekstudiow{Matematyka}
%
% Kierunek studiów po angielsku wybieramy spośród następujących:
% 1) Mathematics
% 2) Mathematics and Statistics
% 3) Applied Mathematics
\kierunekstudiowang{Mathematics}
%
% Specjalność wybieramy spośród następujących: 
% KIERUNEK: Matematyka
% 1) Matematyka teoretyczna,
% 2) Statystyka matematyczna,
% 3) Matematyka finansowa i ubezpieczeniowa,
%
% KIERUNEK: Matematyka i Statystyka
% 4) Matematyka,
% 5) Statystyka i analiza danych, 
%
% 6) -- (w przypadku braku specjalizacji).
\specjalnosc{Matematyka teoretyczna} 
%
% Specjalność w języku angielskim wybieramy spośród następujących:
% KIERUNEK: Matematyka
% 1) Theoretical Mathematics,
% 2) Mathematical Statistics,
% 3) Financial and Actuarial Mathematics,
%
% KIERUNEK: Matematyka i Statystyka
% 4) Mathematics,
% 5) Statistics and Data Analysis,
%
% KIERUNEK: Applied Mathematics
% 6) Financial and Actuarial Mathematics, 
% 7) Mathematics for Industry and Commerce,
% 8) Computational Mathematics,
% 9) Modelling, Simulation and Optimization.
%
% 10) -- (w przypadku braku specjalizacji).
\specjalnoscang{Theoretical Mathematics} 
%
% Krótkie streszczenia po polsku i angielsku
% - nie dłuższe niż 530 znaków.
\streszczenie{Tutaj piszemy krótkie streszczenie pracy (nie powinno być dłuższe niż 530 znaków).}
\streszczenieang{Tutaj piszemy krótkie streszczenie pracy w języku angielskim (nie powinno być dłuższe niż 530 znaków).}
%
% Podajemy najważniejsze słowa kluczowe po polsku i angielsku
% - w obu przypadkach, nie więcej niż 150 znaków.
\slowakluczowe{tutaj podajemy najważniejsze słowa kluczowe (łącznie nie powinny być dłuższe niż 150 znaków).}  
\slowakluczoweang{tutaj podajemy najważniejsze słowa kluczowe w języku angielskim (łącznie nie powinny być dłuższe niż 150 znaków)}
%
%
%%%%%%%%%%%%%%%%%%%%%%%%%%%%%%%%%%%%%%%%%%%%%%%%%%%%%%%%%
% Definicje, lematy, twierdzenia, przykłady i wnioski
% Komendy wywołujące twierdzenia, definicje, itd., 
% czyli 'theorem', 'definition', 'corollary', itd., 
% można zmienić wedle uznania.
\theoremstyle{plain}
\newtheorem{theorem}{Twierdzenie}
\numberwithin{theorem}{chapter}
\newtheorem{lemma}[theorem]{Lemat} 
\newtheorem{corollary}[theorem]{Wniosek}
\newtheorem{fact}[theorem]{Fakt}
\theoremstyle{definition}
\numberwithin{theorem}{chapter}
\newtheorem{definition}[theorem]{Definicja} 
\newtheorem{example}[theorem]{Przykład}
\newtheorem{note}[theorem]{Uwaga}
%%%%%%%%%%%%%%%%%%%%%%%%%%%%%%%%%%%%%%%%%%%%%%%%%%%%%%%%%


%%%%%%%%%%%%%%%%%%%%%%%%%%%%%%%%%%%%%%%%%%%%%%%%%%%%%%%%%
%%%%%%%%%%%%%%%%%%%%%%%%%%%%%%%%%%%%%%%%%%%%%%%%%%%%%%%%%
\begin{document}
\frontmatter
\maketitle
\mainmatter
\tableofcontents
%\listoffigures
%\listoftables

{\backmatter \chapter{Wstęp}}
We wstępie zapowiadamy, o czym będzie praca. Próbujemy zachęcić czytelnika do dalszej lektury, np. krótko informując, dlaczego wybraliśmy właśnie ten temat i co nas w nim zainteresowało.

\chapter{Rozdział pierwszy}
Tabela \ref{tab:przykladowa} przedstawia przykładową tabelę. Do tworzenia tabeli służą m.in. środowiska \texttt{tabular} oraz \texttt{table}. Istnieje możliwość numeracji dwustopniowej, gdzie pierwsza cyfra oznacza numer rozdziału, a druga – kolejny numer tabeli w tym rozdziale. Tytuł powinien znajdować się centralnie nad tabelą, $12$ pkt odstępu od tekstu zasadniczego nad i pod tabelą wraz z tytułem. Jeśli tabela jest cytowana – należy podać centralnie pod tabelą źródło jej pochodzenia, np. opracowanie własne, opracowano na podstawie danych z GUS.
\begin{table}[ht]
\caption{Podstawowa Tabela}
\centering
\begin{tabular}{ccc}
\hline
\hline                       
Państwo & PKB (w milionach USD )& Stopa bezrobocia  \\  [0.5ex] 
\hline 
Stany Zjednoczone & 75 278 049 & 4,60\%  \\
Chiny & 11 218 281 & 4,10\%   \\
Japonia & 4 938 644 & 3,10\%  \\
Niemcy & 3 466 639 & 6,00\%   \\
Wielka Brytania & 2 629 188 & 4,60\%  \\ [1ex]  
\hline 
\end{tabular}
\caption*{\textit{Źródło: opracowanie własne}}
\label{tab:przykladowa} 
\end{table}

Do cytowania używamy komendy \texttt{cite}. W nawiasie klamrowym podajemy klucz, którego użyliśmy w pliku \emph{bibliografia.bib}. Przykład: \cite{einstein} lub \cite[chap. 2]{latexcompanion}.

\section{Podrozdział pierwszy}

\begin{table}[H]
\caption{Podstawowa Tabela}
\centering
\begin{tabular}{ccc}
\hline
\hline                       
Państwo & PKB (w milionach USD )& Stopa bezrobocia  \\  [0.5ex] 
\hline 
Stany Zjednoczone & 75 278 049 & 4,60\%  \\
Chiny & 11 218 281 & 4,10\%   \\
Japonia & 4 938 644 & 3,10\%  \\
Niemcy & 3 466 639 & 6,00\%   \\
Wielka Brytania & 2 629 188 & 4,60\%  \\ [1ex]  
\hline 
\end{tabular}
\caption*{\textit{Źródło: opracowanie własne}}
\label{tab:przykladowa2} 
\end{table}

\section{Podrozdział drugi}

Rysunki do pracy dyplomowej należy wstawiać w sposób podobny do wstawiania tabel, z~zasadniczą różnicą polegającą na tym, że podpis powinno umieszczać się centralnie pod rysunkiem, a nie powyżej niego. Numeracja i sposób cytowania pozostają bez zmian, przy czym tabele i rysunki nie mają numeracji wspólnej, np. po Tabeli \ref{tab:przykladowa2} występuje Rysunek \ref{rys1} (o ile jest to pierwszy rysunek rozdziału pierwszego), a nie Rysunek $1.3$.

\begin{figure}[ht]

\centering
                     
\includegraphics[scale=0.27]{logo_w13.jpg}
\caption{Podstawowy Rysunek}\label{rys1}
\end{figure}
\label{rys:przykladowy} 


% MOJE ================================================
Definicje:
orbita,
uklad dynamiczny,
przestrzen metryczna zwarta,
orbita okresowa
RODZAJE CHAOSU (Devaneya, Li Yorka)
niezmienniczosc zbioru ze wzgledu na odwzorowanie
gęstość
napisac ze chaotycznosc odwzorowania rozumiem przez chaotycznosc odpowiedniego ukladu dynamicznego
zbiór rezydualny
topologia
baza topologii
kula otwarta



\chapter{Twierdzenie o rozszerzaniu odwzorowań chaotycznych w sensie Devaney'a}
\begin{theorem}[O rozszerzaniu]\label{twierdzenie_glowne}
Niech $(X, \rho)$ będzie zwartą przestrzenią metryczną bez punktów izolowanych, oraz niech $f \in C(X)$ będzie odwzorowaniem chaotycznym w sensie Devaney'a. Wówczas odwzorowanie $f$ można rozszerzyć do odwzorowania $F \in C_{\triangle}(X \times I)$ (to znaczy tak, że $f$ jest odwzorowaniem bazowym dla $F$) w taki sposób, że:
\begin{enumerate}[label=(\roman*)]
\item\label{twierdzenie_glowne_a} $F$ jest również chaotyczne w sensie Devaney'a, 
\item\label{twierdzenie_glowne_b} $F$ ma taką samą entropię topologiczną jak $f$, 
\item\label{twierdzenie_glowne_c} zbiory $X \times \{0\}$ i $X \times \{1\}$ są niezmiennicze ze względu na $F$.
\end{enumerate}
\end{theorem}
\cite{balibrea2003topological}

Na potrzeby dowodu wprowadźmy pojęcia odległości między odwzorowaniami. Niech $(M, \sigma)$ będzie zwartą przestrzenią metryczną, rozważmy odwzorowania $h,k \in C(M)$. Odległość między nimi zdefiniujmy jako $\max_{m \in M} \sigma(h(m), k(m))$ i oznaczmy ją jako $d_1(h,k)$.
Odległość między odwzorowaniami trójkątnymi definiujemy wówczas następująco: Niech $(X, \rho)$ i $(Y, \tau)$ będą zwartymi przestrzeniami metrycznymi a $F(x,y) = (f(x), g_x(y))$ i $\Phi(x,y) = (\phi(x), \psi_x(y))$ trójkątnymi odwzorowaniami należącymi do $C_\triangle(X \times Y)$. Odległość definiujemy wówczas jako 
\begin{align*}
d_2(F, \Phi) = \max_{(x,y \in X \times Y} \max\{\rho(f(x),\phi(x)), \tau(g_x(y), \psi_x(y))\} \\ 
= \max\{d_1(f,\phi), \max_{x \in X}d_1(g_x, \psi_x)\}
\end{align*}




% TODO ===============================================================
 kontynuowac

% END TODO ===========================================================

\begin{lemma}\label{nigdziegestosc_orbit}
Niech  $(X, \rho)$ będzie zwartą przestrzenią metryczną bez punktów izolowanych. Każda okresowa orbita $P_0$ odwzorowania $f \in C(X)$ jest nigdziegęstym domkniętym podzbiorem $X$.
\end{lemma}

\begin{lemma}\label{rownosc_entropii_gdy_wlokna_monotoniczne}
% TODO
\end{lemma}

\begin{lemma}
\label{warunki_dostateczne_chaotycznosci_devaneya}
% TODO
dla przestrzeni nieskonczonej i zwartej czyli referencja 25 z pracy glownej
\end{lemma}

\begin{lemma} \label{F_jest_niepustym_domknietym_podzbiorem_trojkatnych}
% TODO
\end{lemma}

\begin{lemma} \label{trojkatne_tworza_przestrzen_metryczna_zupelna}
% TODO
\end{lemma}

\begin{lemma} \label{przekroj_rezydualnych_jest_rezydualny}
% TODO
\end{lemma}

\begin{theorem} \label{tranzytywne_rezydualne_w_F}
% TODO
\cite{alseda1999entropy} - dowod twierdzenia 1.5
\end{theorem}

\begin{lemma} \label{Fso_sa_otwartymi_podzbiorami_F}
% TODO
\end{lemma}



\begin{proof}[Dowód twierdzenia o rozszerzaniu]
Odwzorowanie $f$ jest chaotyczne w sensie Devaney'a, zatem spełnia warunek $(2)$, czyli ma gęsty zbiór punktów okresowych, w szczególności istnieje orbita okresowa.
Możemy zatem ustalić okresową orbitę $P_0$ odwzorowania $f$. Z lematu \ref{nigdziegestosc_orbit} mamy, że $P_0$ jest nigdziegęstym, domkniętym podzbiorem $X$.

Rozważmy zbiór $\mathcal{F}$ wszystkich odwzorowań $F = (f, g_x)$ ze zbioru $C_\triangle(X \times I)$ spełniających następujące warunki:
\begin{enumerate}
\item\label{dowod_glowny_a} Odwzorowanie bazowe $f$ spełnia założenia twierdzenia \ref{twierdzenie_glowne}.
\item\label{dowod_glowny_b} $\forall_{x \in X}$ odwzorowanie $g_x$ jest niemalejące i krańce przedziału I pozostawia niezmienione.
\item\label{dowod_glowny_c} $\forall_{x \in P_0}$ $g_x$ jest identycznością 
\end{enumerate}

Warunek \ref{dowod_glowny1} implikuje, że dla każdego odwzorowania z $\mathcal{F}$ zbiory $X \times \{0\}$ i $X \times \{1\}$ są niezmiennicze, czyli $\forall_{F \in \mathcal{F}}$ zachodzi warunek \ref{twierdzenie_glowne_c}  twierdzenia \ref{twierdzenie_glowne}.
Zachodzenie warunku \ref{twierdzenie_glowne_b} twierdzenia \ref{twierdzenie_glowne} dla każdego odwzorowania $F \in \mathcal{F}$ wynika z lematu \ref{rownosc_entropii_gdy_wlokna_monotoniczne}.
Pozostaje zatem wykazać prawdziwość warunku \ref{twierdzenie_glowne_a}, czyli chaotyczność w sensie Devaney'a jakiegoś odwzorowania $F \in \mathcal{F}$. Takie odwzorowanie będzie bowiem łącznie spełniało wszystkie 3 warunki, czyli tezę twierdzenia.

Z lematu \ref{warunki_dostateczne_chaotycznosci_devaneya} wynika, że aby odwzorowanie $F$ było chaotyczne w sense Devaneya potrzeba i wystarcza, żeby spełniało dwa poniższe warunki:
\begin{enumerate}
\item \label{devaney_pierwsza_wlasnosc} $F$ jest topologicznie tranzytywne
\item \label{devaney_druga_wlasnosc} zbiór punktów okresowych odwzorowania $F$ jest gęsty w $(X \times I)$
\end{enumerate}
Jest tak gdyż przestrzeń $(X \times I)$ spełnia założenia lematu, tj. jest przestrzenią nieskończoną i zwartą.

Chcemy wykazać, że $\exists_{F \in \mathcal{F}}$ będące jednocześnie topologicznie tranzytywne i posiadające gęsty w $(X \times I)$ zbiór punktów okresowych. Z lematu \ref{F_jest_niepustym_domknietym_podzbiorem_trojkatnych}wiemy, że$\mathcal{F}$ jest niepustym, domkniętym podzbiorem ptrzestrzeni $C_{\triangle}(X \times I)$, która jak wynika z lematu \ref{trojkatne_tworza_przestrzen_metryczna_zupelna} jest przestrzenią metryczną zupełną. 

Przekrój dwóch zbiorów rezydualnych jest rezydualny (patrz lemat \ref{przekroj_rezydualnych_jest_rezydualny}) a więc niepusty. Wystarczy zatem pokazać, że oba zbiory:
\begin{itemize}
\item zbiór odwzorowań topologicznie tranzytywnych
\item zbiór odwzorowań, których zbiór punktów okresowych jest gęsty w $(X \times I)$
\end{itemize}
są rezydualne w $\mathcal{F}$. Wówczas każde odwzorowanie należące do ich przekroju będzie spełniało wszystkie trzy warunki tezy twierdzenia \ref{twierdzenie_glowne} 

Zbiór odwzorowań tranzytywnych jest rezydualny w $\mathcal{F}$, zostało to udowodnione w twierdzeniu \ref{tranzytywne_rezydualne_w_F}

Pozostało wykazać, że również zbiór odwzorowań posiadających gęsty zbiór punktów okresowych jest rezydualny w $\mathcal{F}$.
Oznaczmy zbiór takich odwzorowań (jednocześnie należących do $\mathcal{F}$) przez $\mathcal{F}_{DP}$.

Niech $\{U_i^X\}_{i=1}^{\infty}$ będzie bazą topologii $X$ i niech $\{U_i^I\}_{i=1}^{\infty}$ będzie zbiorem wszystkich odcinków otwartych o końcach wymiernych, należących do odcinka otwartego $(0, 1)$. Niech $\{U_i\}_{i=1}^{\infty}$ będzie ponumerowaniem zbioru $\{U_i^X \times U_j^I : i,j \in \mathbb{N}\}$. Wtedy każda kula otwarta w $X \times I$ zawiera jakiś spośród otwartych zbiorów $U_i$.

Dla każdego $i=1,2,...$ niech zbiór $\mathcal{F}_{SO}^i$ ("S" - stabilny, "O" - okresowy) będzie zdefiniowany następująco. Odwzorowanie G należy do $\mathcal{F}_{SO}^i$ wtedy i tylko wtedy, gdy należy do $\mathcal{F}$, posiada punkt okresowy w $U_i$ oraz wszystkie dostatecznie bliskie $G$ odwzorowania z $\mathcal{F}$ również posiadają punkt okresowy w $U_i$ (być może różne od punktuów okresowych odwzorowania $G$). Zbiory $\mathcal{F}_{SO}^i$ są otwartymi podzbiorami $\mathcal{F}$ (patrz lemat \ref{Fso_sa_otwartymi_podzbiorami_F}). Ponieważ $\mathcal{F}_{DP} \supseteq \bigcap_{i=1}^{\infty} \mathcal{F}_{SO}^i$ aby pokazać, że $\mathcal{F}_{DP}$ jest rezydualny w $\mathcal{F}$ wystarczy pokazać, że $\forall_{i \in \mathbb{N}} \mathcal{F}_{SO}^i$ jest gęsty w $\mathcal{F}$. (JEst tak, ponieważ TODO)



\end{proof}




% END MOJE ====================================================

\chapter{Definicje, lematy, twierdzenia, przykłady i wnioski}
Definicje, lematy, twierdzenia, przykłady i wnioski piszemy w pracy tak:
\begin{definition}[Martyngał]
Tu piszemy treść definicji martyngału.
\end{definition}
\begin{lemma}[]% w nawiasie kwadratowym można napisać jego nazwę
Tu piszemy treść lematu.
\end{lemma}

{\backmatter \chapter{Podsumowanie}}
Podsumowanie w pracach matematycznych nie jest obligatoryjne. Warto jednak na zakończenie krótko napisać, co udało nam się zrobić w pracy, a czasem także o tym, czego nie udało się zrobić.

{\backmatter \chapter{Dodatek}}
Dodatek w pracach matematycznych również nie jest wymagany. Można w nim przedstawić np. jakiś dłuższy dowód, który z pewnych przyczyn pominęliśmy we właściwej części pracy lub (np. w przypadku prac statystycznych) umieścić dane, które analizowaliśmy.

%%%%%%%%%%%%%%%%%%%%%%%%%%%%%%%%%%%%%%%%%%%%%%%%%%%%%%%%%
% BIBLIOGRAFIA
% W tworzeniu bibliografii najlepiej korzystać z BibTex'a, 
% który jest częścią systemu Tex. W naszym przypadku funkcję 
% przechowalni literatury, do której się odwołujemy, pełni 
% plik bibliografia.bib. Nie musimy ręcznie dodawać nowych 
% pozycji do bibliografii. Możemy wejść np. na stronę 
% https://mathscinet.ams.org/mathscinet/index.html, 
% znaleźć odpowiednią pozycję, wybrać ją, a następnie zmienić 
% 'Select alternative format' na BibTeX, skopiować uzyskany 
% tekst, wkleić do pliku bibliografia.bib i skompilować. 
% Gotowe informacje do pliku bibliografia.bib można znaleźć 
% także na https://arxiv.org - gdy znajdziemy interesującą nas 
% pracę, szukamy 'References & Citations' i klikamy 'NASA ADS', 
% a potem 'Bibtex entry for this abstract' 
% i postępujemy tak jak wcześniej.
%%%%%%%%%%%%%%%%%%%%%%%%%%%%%%%%%%%%%%%%%%%%%%%%%%%%%%%%%
\newpage
% w nawiasie klamrowym wpisujemy nazwę pliku z bibliografią w formacie .bib
\bibliografia{bibliografia.bib} 
\end{document}